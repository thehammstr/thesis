% !TEX root = ../thesis.tex

\chapter{Mathematical Notation}
\label{ch.Notation}
\section{Notation}

Before developing the equations, the notation used in this dissertaion is as follows:

\emph{Position vectors} are written as
\begin{equation*}
 \vec{r}^{~\nicefrac{A}{B}} 
\end{equation*}
where $A$ is the ``to" point and $B$ is the ``from" point.

\emph{Unit vectors} are denoted with a hat ($~\hat{}~$), and scalars are unadorned. Note: a hat can also indicate that the quantity is estimated. The meaning should be clear from context, especially when both hat and vector arrow are present.

% Velocity
\emph{Velocities} are written as 
\begin{equation*} 
~^{A}\vec{v}^{P} 
\end{equation*}
which should be read as ``the velocity of point $P$ in reference frame $A$."

% Angular Velocity
\emph{Angular velocities} are written as 
\begin{equation*}
 ~^{A}\vec{\omega}^{B}
\end{equation*}
which should be read as ``the angular velocity of reference frame $B$ in reference frame $A$."

Vector \emph{bases} will be denoted with right subscripts. For example $^I\vec{v}^P_{Q}$ is the velocity of point $P$ through reference frame $I$, \emph{expressed in} reference frame Q.   
% Derivatives
When taking \emph{derivatives}, left superscripts refer to the reference frame in which the derivative is being taken.
\begin{equation*}
 \frac{^{A}d(~)}{~dt}
\end{equation*}

Points are defined as:
\begin{align*}
O =&~ \text{Origin, fixed in inertial frame}\\
V_{cm} =&~ \text{Vehicle (Collocated with DVL)} \\
B_{cm} =&~ \text{Iceberg center of mass}\\
S_i =&~ \text{Sounding location (fixed in iceberg frame)}
\end{align*}
\\
\\
Reference frames are denoted using:
\begin{align*}
I =&~ \text{Inertial}\\
V =&~ \text{Vehicle} \\
B =&~ \text{Iceberg}
\end{align*}

%%%%%%%%%%%%%%%%%%
% end vector definitions
%%%%%%%%%%%%%%%%%%
$\theta_c$
% !TEX root = ../Thesis.tex

\chapter{Solving for Iceberg Shape using GraphSLAM}
\label{ch.GraphSLAM}

Chapter \ref{ch.IcebergGeometry} recast the problem of iceberg mapping in familiar terms. The vehicle's trajectory through the iceberg frame of reference can be dead-reckoned based on the DVL measurements, and the iceberg's rotation is can be modeled as a drifting rate gyro bias or low-precision compass, both of which have been dealt with extensively in the robotics community.

However, the elimination of the inner loop optimization carries the implicit assumption that any error comes entirely from the iceberg's rotation. In reality, there will be other errors due to measurement and odometry noise that should be considered alongside the rotational error to produce a map of the highest quality possible. GraphSLAM provides a theoretical framework that allows the error associated with each quantity to be weighted according to its uncertainty relative to all other measurements. Additionally, the structure used by GraphSLAM to represent the mapping problem allows it be used to solve large mapping problems very efficiently. For an illustration of this, Montemerlo et. al. calculated a map of Stanford's Main Quad, including $10^5$ poses and $10^8$ measurements in roughly 30 seconds of CPU time \cite{?}. As discussed in Chapter \ref{ch.RelatedWork}, many SLAM implementations exist, each with its advantages and disadvantages. In this instance, the ease of representing the iceberg mapping problem as a pose graph, the ability to solve large optimizations efficiently, and the fact that the calculations did not need to be performed online drove the selection of GraphSLAM as a solution method.

\section{GraphSLAM overview}

\begin{figure}[htbp]
   \centering
   \includegraphics[width=.7\textwidth]{../graphics/factor_graph.png} % requires the graphicx package
   \caption{GraphSLAM models the mapping problem as a probabilistic graph.  The poses $x_i$ and $x_{i+1}$ are linked via odometry terms. Additionally, any two poses where the same landmark was observed are linked by measurement terms through that landmark. These links represent Gaussian conditional probabilities. \emph{PLACEHOLDER GRAPHIC}}
   \label{fig:GraphSLAM}
\end{figure}


\begin{figure}[htbp]
   \centering
   \includegraphics[width=\textwidth]{../graphics/InfoMatrix.png} % requires the graphicx package
   \caption{Internal representation of the GraphSLAM problem. Pose and map state relationships are encoded in the sparse, symmetric positive-semidefinite matrix, $\Omega$. The structure of the matrix allows efficient solution via the method of conjugate gradients. Additionally, the cumulative effect of multiple obeservations of landmarks can be marginalized out, resulting in a lossless reduction of the optimization, and further improving speed. \emph{From Probabilistic Robotics, by Thrun, Burgard, and Fox.}}
   \label{fig:GraphSLAMinternal}
\end{figure}

\subsection{GraphSLAM as nonlinear least squares}

GraphSLAM represents the mapping problem as a probabilistic graph. Pose nodes correspond to the vehicle's state at discrete times $t$. If an explicit map is to be used, other nodes represent map feature locations. 

Consecutive  pose nodes are linked through odometry links. These links encode the conditional probability distribution with Gaussian form

\begin{equation}
p\left(x_{t+1}~|~x_t,u_t\right) = \eta~ exp\left(-\frac{1}{2}\left[x_{t+1} - g\left(u_t,x_t\right)\right]^{\intercal}R_t^{-1}\left[x_{t+1} - g\left(u_t,x_t\right)\right]\right)
\end{equation}

where $g\left(u_t,x_t\right)$ is the motion update equation, a function of state $x_t$ and control input $u_t$ at time $t$, and $R_t$ is the process noise covariance.

Map nodes are connected to the pose nodes from which they were observed through measurement links. These take the same form of the odometry links:

\begin{equation}
p\left(z^i_{t}~|~x_t,m_j\right) = \eta~ exp\left(-\frac{1}{2}\left[z^i_{t} - h\left(x_t,m_j\right)\right]^{\intercal}Q_t^{-1}\left[z^i_{t} - h\left(x_t,m_j\right)\right]\right)
\end{equation}

that is, the Normal probability distribution of making measurement $z^i_t$ of map feature $m_j$ at time $t$ given the estimates of the location of $m_j$ and pose $x_t$.

Taking the log of the full joint probability yields a sum of quadratic terms:

\begin{multline}
J = ~x_0^{\intercal}\Omega_0 x_0 ~+
\sum_{t}{\left[x_{t+1} - g\left(u_t,x_t\right)\right]^{\intercal}R_t^{-1}\left[x_{t+1} - g\left(u_t,x_t\right)\right]} ~+ \\ \sum_{t}{\sum_{i}{\left[z^i_{t} - h\left(x_t,m_j\right)\right]^{\intercal}Q_t^{-1}\left[z^i_{t} - h\left(x_t,m_j\right)\right]}}
\label{eq.quadcost}
\end{multline}

The first term is necessary to ``anchor" the map, since all other constraints are purely relative between poses. In effect, it fixes the first point to the origin. 

The maximum likelihood estimates for all free parameters in the optimization are those that minimie this cost function, $J$.

Internally, these links are linearized, then encoded in an information matrix and an information vector, as shown in Figure \ref{fig:GraphSLAM} and fully developed in \cite{ref?}. The information matrix is sparse, and symmetric positive semi-definite. The sparsity allows the measurements to be marginalized out, creating a much smaller matrix defined only in the pose space, but containing all the information from the map features. This reduction in size and positive semi-definite structure allows the use of fast solution algorithms like the conjugate gradient method. 

\subsection{Minimum spring energy analogy}

A useful analogy for gaining intuition into how the GraphSLAM solver works uses the concept of spring potential energy. 

\begin{align}
P_{\text{spring}} &= \frac{1}{2}k\left(l-l_n\right)^2
\label{eq.spring}
\end{align}

Note the similarity between equation \ref{eq.spring} and the right hand side terms of equation \ref{eq.quadcost}. All of them are weighted quadratic functions.

In this analogy, each graph link is modeled as a spring separating the nodes, as shown in Figure \ref{fig:springs}. The natural length $l_n$ of the spring corresponds to the mean of the conditional probability,  $g\left(u_t,x_t\right)$ for odometry, and $h\left(x_t,m_j\right)$ for measurements. The stiffness $k$ is inversely proportional the covariance $R$ or $Q$: the more certain the measurement, the stiffer the effective spring. 

Once the nodes are all tied together with springs of appropriate length and stiffness, any deviation from the natural length will result in additional spring potential energy. The GraphSLAM solution is identical to the \emph{minimum energy} state of this system. In fact, if the nodes were modeled as masses, and viscous damping were applied, dynamic motion simulation could be used to compute the same solution as GraphSLAM. The system would ``relax" into its minimum energy state, and the steady state dynamic solution would be the GraphSLAM MLE solution. While such a method would likely be much slower than using the actual GraphSLAM solution algorithm, the idea of modeling the system as a collection of masses, springs, and dampers provides useful intuition on how the algorithm manipulates the parameters to arrive at a solution.

\begin{figure}[htbp]
   \centering
   \includegraphics[width=.65\textwidth]{../graphics/springs.png} % requires the graphicx package
   \caption{GraphSLAM is equivalent to the minimum energy configuration of a mass-spring-damper system. Each spring's natural length and stiffness are determined by the corresponding measurement and measurement covariance, respectively.}
   \label{fig:springs}
\end{figure}

%\Omega = \left[\begin{array}{c c c c c | c c c c}
%u_{1,1} & u_{1,2} &     0   & 0 & 0 &  z_{1,1} & \hdots & z_{1,i} & 0 \\
%u_{1,2} & u_{2,2} & \ddots & 0 & 0 &  z_{2,1} & \hdots & z_{2,j} & 0 \\
%0       & \ddots & \ddots &     \ddots  &   0    &    \ddots     &    \ddots   &    \ddots & 0    \\
%0 & 0 & \ddots & u_{n-1,n-1} & u_{n,n-1} & 0 & \ddots & \ddots & \vdots \\
%0 & 0 & 0 & u_{n,n-1} & u_{n,n} & 0 & \hdots & z_{n,m-1}  & z_{n,m}\\
%\hline 
%z_{1,1} & z_{2,1} & 0 & 0 & 0 & l_{1,1} & 0 & 0 & 0\\
%\vdots & \vdots & 0 & 0 & z_{n,m-1} & 0 & \ddots & 0 & 0 \\
%\vdots & \vdots & 0 & 0 & z_{n,m-1} & 0 & 0 & \ddots & 0 \\
%\vdots & \vdots & 0 & 0 & z_{n,m} & 0 & 0 & 0 & l_{m,m} 
%\end{array}\right]
%, ~\xi = \left[\begin{array}{c}
%p_1 \\
%p_2 \\
%\vdots \\
%p_{n-1} \\
%p_n \\
%\hline
%m_1 \\
%m_2 \\
%\vdots \\
%m_m
%\end{array}\right]

\section{State and map representation}

GraphSLAM can be implemented with either explicit or implicit map representations In the explicit case, the locations of discrete features in the map appear as free parameters in the optimization. Additional information about the appearance of the feature is also often included. The locations of these features are estimated in a probabilistic framework simultaneously with the vehicle location. Mutual information regarding navigation offset between two poses that observe the same feature is encoded through that feature. 

The other option is to use an \emph{implicit} map representation, in which information from the terrain is used to estimate the robot's trajectory, but th

The following sections will discuss these different approaches to map representation as they pertain to iceberg mapping. They will also demonstrate the mathematical equivalency between the two approaches. In general, the type of sensor, and terrain will dictate which approach is more convenient. The results presented in Chapter \ref{ch.Results} use an implicit map representation, but an explicit map may be better suited to future work in feature-based automated loop closure as discussed in Chapter \ref{ch.LoopClosure}.

\subsection{Explicit map representation}

In the explicit map representation variant of GraphSLAM, the location of features within the map are solved explicitly alonside the vehicle trajectory. This is often used when the map consists primarily of discrete features, such as passive beacons \cite{?}, trees or pilings \cite{Langellaan}, or robust image features \cite{Augenstein}. 


\subsection{Implicit map representation}

The implicit map representation uses information in terrain viewed multiple times, but encodes that information directly in pose-to-pose constraints. In effect, this jumps directly to the last graphic in Figure \ref{fig:GraphSLAMinternal}. This is useful when the information in the terrain is dispersed throughout, as opposed to being concentrated in well-localized recognizable landmarks. This is often the case when the vehicle navigates through natural terrain using a range scanner with a wide field of view. In these scenarios, larger scale correlation may provide better offset data than local features. 


\section{Known vs. unknown correspondence}

\section{Summary}

GraphSLAM is a method well-suited to the problem of mapping icebergs. Its structure 

% !TEX root = ../thesis.tex

\chapter{Ground-Based Experiments}
\label{ch.GroundBasedExperiments}

Blargh blargh.
\section{Introduction}
\label{sec.Introduction}
Meteoroid impact damage on a spacecraft can result from both mechanical and electrical effects.  In this paper, we refer to all small, solid, extraterrestrial objects as meteoroids, with no lower limit on size.  While the mechanical damage resulting from a hypervelocity impact has been well-studied, the electrical effects are less well-understood.  In experiments performed at the Max Planck Institute for Nuclear Physics (MPIK), we used a Van de Graaff dust accelerator to simulate meteoroid impact events on different metallic targets and to characterize the impact using a suite of RF and plasma diagnostic sensors.  Unlike most experiments characterizing impact plasma, we include direct measurements of a freely-expanding plasma from both charged and uncharged targets.  Previous studies used an accelerating grid at the target to direct the impact plasma into a detector.  This paper reports initial results from the retarding potential analyzer (RPA) suite used to characterize the expanding impact plasma.

The remainder of this section provides an overview of the plasma expansion theory motivating this study, examples of hypervelocity impact measurements made in space, and ground-based testing previously performed.  Section \ref{sec.ExperimentalApproach} describes the experiments performed, section \ref{sec.Results} presents the results from the RPA suite, and section \ref{sec.Conclusion} concludes and outlines our plan for further study.

\subsection{Impact Theory}
The plasma expansion model leading to RF emission is described in detail by \citet{Close2010}  The general concept is outlined in figure \ref{fig.ImpactPlasmaExpansion}.  Initially, a particle impacts a surface at meteoroid speeds.  The kinetic energy of the particle is partially converted to vaporization and ionization energy.  This results in a small dense plasma formed by ionization of material from both the particle and the target.  As the plasma expands it quickly becomes collision-free by dilution.  Because of their higher mobility, electrons expand outward faster than ions, creating an ambipolar electric field which pulls the electrons back toward the ions.  As the ions expand out more slowly, the electrons oscillate about the ion front due to the ambipolar electric field, radiating at the plasma frequency.  This frequency decreases with density as the plasma expands at the isothermal sound speed.  The resulting RF signal, which results from coherent electron oscillations, is a chirp with the initial frequency dependent on the initial ion density and the rate dependent on density falloff.  We believe that this model provides a potential mechanism for spacecraft electrical anomalies and failures due to meteoroid impact.

\begin{figure}[\floatplace]
\begin{center}
\includegraphics[width=\textwidth]{ImpactPlasmaExpansion}
\end{center}
\caption{\label{fig.ImpactPlasmaExpansion} Depiction of the plasma generation and expansion process due to hypervelocity impact.}
\end{figure}

\subsection{In-Situ Evidence}
Evidence of electrical effects on spacecraft stem from reported anomalies experienced during meteoroid showers and also from RF measurements on several deep space probes.

At least two spacecraft have been known to experience electrical anomalies associated with meteoroid impact.  In 1993, the Olympus spacecraft in geostationary orbit experienced several concurrent anomalies, including a gyroscope shutdown, during the peak of the Perseid meteoroid shower, leading to loss of attitude control and eventual loss of mission \cite{Caswell1995}.  While there was no direct measurement of an impact event (through momentum transfer), the investigation into the anomalies suggested that a meteoroid impact may have triggered an electrical discharge.  In 2009, the Landsat 5 spacecraft in low Earth orbit also experienced an attitude anomaly involving a gyroscope during the Perseid shower \cite{USGS2009}.  In this case, attitude control and regular operation of the satellite was restored.  For both spacecraft, it was clear that the anomalies were not due to mechanical damage, since the gyroscopes were returned to operation.

The Cassini spacecraft detected impact events from nanometer-sized particles which have been accelerated by the solar wind to 450~km/s, and from micrometer-sized particles from Saturn's rings moving at roughly 10~km/s relative to the spacecraft.  Waveforms were recorded from the 10~m dipole antennas on the Radio and Plasma Wave Science (RPWS) instrument \cite{Kurth2005, Meyer-Vernet2009a}.

The STEREO pair of spacecraft have also detected electrical effects of impact events using the STEREO wave instrument (S/WAVES), which are coincident with optical measurements made by the two spacecrafts' SECCHI instrument suite. \cite{Meyer-Vernet2009b, St.Cyr2009}.  Additionally, many electrical anomalies with an ``unknown catalyst'' on-orbit may be attributed to meteoroid impacts.

\subsection{Ground-Based Testing}
Many studies have been performed at hypervelocity impact facilities including Van de Graaff dust accelerators, light gas guns, rail guns, and plasma drag accelerators.  There has been no complete analysis of the phenomenon of electromagnetic emission from hypervelocity impact, but several previously-reported experiments are particularly relevant to the study of impact-induced electrical effects.

Microwave emission has been measured in light gas gun and rail gun experiments by Takano et al. \cite{Takano2002,Maki2004,Ohnishi2005,Ohnishi2007}  They primarily used 1~g polycarbonate projectiles fired at speeds of 1~km/s to 4~km/s into metal plates.  Microwave signals were picked up by 2~GHz and 22~GHz horn antennas placed outside the 350~mm diameter vacuum chamber near a window, and appeared to be a random sequence of short pulses.

\citet{Starks2006} performed a similar experiment seeking RF emission but reported inconclusive results due to extraneous signals from projectile charging as well as from exhaust plasma from the accelerator.  This experiment used 900~g titanium flyer plates fired at about 12~km/s through indium foil targets.  Unlike the Takano experiments, their RF measurements were taken using DC plates and broadband microwave horns placed inside the chamber behind the foil target.

Crawford and Schultz \cite{Crawford1993,Crawford1999} used the two-stage hydrogen light gas gun at the Ames Vertical Gun Range to shoot 0.16~g to 0.37~g aluminum spheres into powdered dolomite targets.  Impact speeds ranged from 5~km/s to 7~km/s at incidence angles ranging from $15^\circ$ to $90^\circ$.  Magnetic search coils were used to detect magnetic fields generated by impact plasmas and charge detection plates were used to measure electric current as a function of angle and time.  This research is unique in that the setup consists of a powdered target, whereas most other hypervelocity impact tests are performed on solid metallic plates.

\citet{Burchell1996} used the 2 MV Van de Graaff accelerator at the University of Kent to fire $10^{-10}$~g to $10^{-14}$~g iron particles at aluminum and molybdenum semi-infinite targets in a chamber evacuated to $10^{-6}$~mbar.  Impact speeds were 1~km/s to 42~km/s at an incidence angle of $36^\circ$.  The targets were charged to 1~kV.  Ions were measured using an electron multiplier tube positioned at $90^\circ$ to the target while a 1.4~kV photomultiplier tube measured light flash at various other angles of incidence.  The energy of the light flash, normalized to projectile mass, was found to be proportional to speed to a power from 3.0 to 3.6 depending on material.  They concluded based on the 1 $\mu$s duration of the light flashes that the emission was either from ejecta or the impact plasma, and that it was not due to plasma recombination since it was unaffected by simultaneous plasma measurements using a charged grid.  They also claimed based on the measured energy of the plasma that radiant emission is decoupled from its internal energy.

\citet{Putzar2008} have studied the effect of impacts on spacecraft electrical harnesses, but focuses on mechanical damage due to penetration of the spacecraft wall.  Their experiment used a light gas gun to fire aluminum spheres with diameters between 1.5~mm and 4.0~mm at speeds of 6.4~km/s to 7.7~km/s.  The target included power, data, and RF cable harnesses mounted behind multi-layer insulation.  The reported anomalies and failures in the cables were attributed to physical damage, as all impacts resulted in penetration of the structure wall.

There is much disagreement in the field about the mechanisms behind impact-induced radiation.  \citet{Starks2006} attributed impact light flashes to rapid recombination of a fully-ionized plasma, while \citet{Burchell1996} concluded that light flashes are not a result of recombination since they were not affected by his direct plasma measurements, which inhibit recombination.  Takano's research group \cite{Takano2002,Maki2004,Ohnishi2005,Ohnishi2007} associated their microwave signals to microcracking while Starks et al. searched for microwave signals they attributed to plasma oscillation.  Crawford and Schultz \cite{Crawford1993,Crawford1999} posited a macroscopic charge separation, which may be due to their use of a powdered dolomite target rather than the more commonly studied solid metallic targets.

The disparate test environments that were used is a significant factor in comparing the results from different research groups.  In particular, the achievable vacuum levels at light-gas gun facilities (${\sim}0.1$~mbar) are significantly poorer than at Van de Graaff dust accelerators (${\sim}10^{-5}$~mbar).  The effect of ambient atmosphere has been studied for expanding plasma plumes from laser ablation studies \cite{Harilal2003} and is significant for impact-generated plasmas as well.  Most notably, the effect of a collisional ambient atmosphere will suppress free expansion and hence coherent oscillations in the plasma.

\section{Experimental Approach}
\label{sec.ExperimentalApproach}
Hypervelocity impact experiments were performed at the Max Planck Institute for Nuclear Physics (MPIK).  The goal of these experiments was to detect RF emission and to characterize the expanding plasma from impacts using a Van de Graaff dust accelerator.

\subsection{Facility}
The 2~MV Van de Graaff dust accelerator at MPIK is functionally similar to the one described in detail by \citet{Burchell1999}.  It accelerates positively charged particles from the 2~MV terminal through a series of potential coils and into a vacuum chamber.  The speed of the particles is limited by the amount of surface charge that can be accumulated.  Since mass drops off faster than surface area, smaller particles tend to be accelerated to higher speeds.  A particle selection unit uses induction tubes to measure the charge and speed of particles and capacitor plates to deflect particles not meeting a selection threshold.  The mass of particles can be determined from the measured charge and speed by equating the kinetic energy of the particle to the potential energy across the Van de Graaff terminals:
\begin{equation}
\label{eq.VdGEnergyBalance}
\frac{1}{2}mv^2 = qU,
\end{equation}
where $m$, $v$, and $q$ are the mass, speed, and charge of the particle and $U$ is the accelerating voltage.

The vacuum chamber is 1.4~m in diameter and can be depressurized to $10^{-6}$~mbar.  Experiments were performed at pressures between $2.5\times10^{-6}$~mbar and $8.1\times10^{-6}$~mbar, which corresponds to a mean free path greater than the size of the chamber.  The particle selection unit generates programmable location pulses that are used as trigger signals when the particle is predicted to be at a specified distance downstream from the accelerator.

Spherical iron projectiles were used throughout the experiment.  For the impact events analyzed in this paper, the particles span a mass range from $10^{-15}$~g to $10^{-10}$~g and a speed range from 2~km/s to 45~km/s.  The particle data set and associated cumulative distributions are shown in figures \ref{fig.ParticleDist} and \ref{fig.ParticleCumDist}.

\begin{figure}[\floatplace]
\begin{center}
\includegraphics[width=0.5\textwidth]{RPA_mv_scatter}
\end{center}
\caption{\label{fig.ParticleDist} Distribution of particle masses and speeds.  Lower-speed particles are achievable but were deflected by the particle selection unit.}
\end{figure}

\begin{figure}[\floatplace]
\begin{center}
\includegraphics[width=0.45\textwidth]{RPA_m_cum_dist}
\includegraphics[width=0.45\textwidth]{RPA_v_cum_dist}
\end{center}
\caption{\label{fig.ParticleCumDist} Left: Cumulative distribution of particle masses. Right: Cumulative distribution of particle speeds.}
\end{figure}

\subsection{Targets}
\label{sec.Targets}
Five different targets were used of varying material and thickness as described in table \ref{tab.Targets} and shown in figure \ref{fig.Targets}.  Except for the aluminum foil target, the other targets are all thick enough to be considered semi-infinite with respect to the projectiles.  The brass and copper targets doubled as stub antennas to measure the E-field at the point of impact.  The tungsten and two aluminum targets were capable of simulating spacecraft charging using an RC circuit connected to a $\pm1$~kV voltage source.  This RC circuit is composed of a 435~nF capacitor from the target to chamber ground, and a 225~k$\Omega$ resistor feeding the capacitor from the voltage source.

The targets were mounted in the chamber at a pitch of $30^\circ$ up from the horizontal beam line so that the sensors could be aligned more closely with the target normal vector.  A mechanical feedthrough in the chamber allowed for lateral translation of the target platform to shift each target into the beam line without opening the chamber.

\begin{table}[\floatplace]
\begin{center}
\begin{tabular}{cccc}
 & Material & Thickness & Geometry \\
 \hline
1 & Brass & 1~mm & 8~mm diameter circle \\
2 & Copper & 0.5~mm & 40~mm diameter circle\\
3 & Tungsten & 0.0508~mm & 24~mm $\times$ 26~mm\\
4 & Aluminum & 2.70~mm & 29~mm $\times$ 29~mm\\
5 & Aluminum foil & 0.0127~mm & 43~mm $\times$ 16~mm (window)
\end{tabular}
\end{center}
\caption{\label{tab.Targets} Overview of key target parameters.}
\end{table}

\begin{figure}[\floatplace]
\begin{center}
\includegraphics[width=0.3\textwidth]{TargetBrass}
\includegraphics[width=0.3\textwidth]{TargetCopper}
\includegraphics[width=0.3\textwidth]{TargetMulti}
\end{center}
\caption{\label{fig.Targets} Left: Brass target. Center: Copper target. Right: Tungsten and aluminum targets.}
\end{figure}

\subsection{Sensors}
A suite of plasma diagnostic and RF sensors was used to characterize the expanding plasma plume and any emitted RF signal.  Results from the RF sensor suite are beyond the scope of this paper but a brief description of the sensors is presented for completeness.  The sensor layout is depicted in figure \ref{fig.ChamberGeometry}.

\begin{figure}[\floatplace]
\begin{center}
\includegraphics[width=0.45\textwidth]{ChamberGeometry}
\includegraphics[width=0.45\textwidth]{RPAGeometry}
\end{center}
\caption{\label{fig.ChamberGeometry} Left: Layout of the targets and sensors in the vacuum chamber. Right: View looking upward at the targets and RPAs from below.}
\end{figure}

\subsubsection{RF sensors}
The RF sensor suite is composed of the E-field sensor described above in section \ref{sec.Targets}, two log periodic arrays, and three VLF loop antennas.  The E-field sensor includes a built-in amplifier stage at the target and two low-noise amplifier stages outside the chamber.  The log periodic arrays were arranged to capture horizontally and vertically polarized signals.  Each array had two low-noise amplifier stages outside the chamber.  The three VLF loops were collocated and oriented to measure the magnetic field in three orthogonal directions.

\subsubsection{Plasma sensors}
Two retarding potential analyzers (RPAs) were positioned near the impact point to measure the expanding plasma plume.  The design of the RPA is similar to that described by \citet{Marrese1997} and \citet{Heelis1998}.  Each has an effective planar collecting area of 10~cm$^2$ behind a sequence of four grids and supported by amplifier electronics.  The four grids, from front to back, are the:
\begin{enumerate}
\item Floating grid, to shield the internal fields from the incoming plasma;
\item Repeller grid, to select for ions or electrons;
\item Threshold grid, to repel low-energy particles of the selected species; and
\item Suppressor grid, to prevent escape of electrons due to secondary ionization.
\end{enumerate}
For these experiments, the grid voltages were kept low (less than 5~V) to measure the plasma freely impinging on the collecting area.  The data show no dependence on grid voltage variations at this level.

The collecting area is connected first to a transimpedance amplifier stage and then to a differential amplifier stage within the RPA housing.  This provides a differential signal that is brought out of the chamber on a twisted pair cable.  The differential signal is sampled on two separate data acquisition channels.  The amplifier electronics and floating grid are grounded to the RPA housing, but kept isolated from the external chamber ground.  The transimpedance gain of the RPA is 16~mV/nA, as determined by a comparison of actual and simulated responses to a test input signal.

The RPAs were placed 75~mm and 150~mm from the impact point (measured to the floating grid), and tilted $15^\circ$ and $30^\circ$ off the target normal vector, respectively, as shown in figure \ref{fig.ChamberGeometry}.  This configuration minimized the angles between the RPAs and the target normal vector, subject to the physical constraints of the RPA housings.  The plasma plume is expected to expand primarily along the target normal vector, so the RPAs were arranged to maximize the expected signal.  The two RPAs were placed at different distances from the impact point in order to yield a measurement of plasma expansion speed.  In the remainder of this paper we will designate the nearer RPA as RPA-A and the further one as RPA-B.

\section{Results}
\label{sec.Results}
A total of 2913 events were recorded where the particle mass and speed could be determined from an inductive pickup loop.  From this data set, 1247 events showed a signal peak (evidence of plasma detection) from RPA-A and 92 showed a signal peak from both RPAs using a matched filter.  Table \ref{tab.EventNumbers} summarizes the number of events studied by target material and bias.  We present results on the dependence of plasma signal on target material and bias, the signal strength as a function of projectile parameters, and an estimate of impact plasma expansion speed.

\begin{table}[\floatplace]
\begin{center}
\begin{tabular}{cc|ccc}
&&\multicolumn{3}{|c}{Material}\\
&&W&Al&Al foil\\
\hline
\multirow{3}{*}{\begin{sideways}Bias\end{sideways}}&+1~kV&460&215&209\\
  &Float&628&224&225\\
&-1~kV&300&293&188
\end{tabular}
\end{center}
\caption{\label{tab.EventNumbers} 2742 events broken down by target material and bias.  A further 171 events occurred on the brass and copper targets, but did not yield a large enough sample of detectable RPA signals to be statistically significant.}
\end{table}

\subsection{Plasma Signal}
Weak signals were identified using a matched filter with a synthetic impulse response.  This impulse response was constructed by fitting a decaying first-order exponential to a strong signal identified from RPA-A.  The ``canonical'' signal pulse and the synthetic impulse response are shown in figure \ref{fig.RPA6322withHsynth}.  The RPA signal from events with large and small signals are shown in figures \ref{fig.RPAQ6614} and \ref{fig.RPAQ8634}, respectively, with their associated matched filter signal.

\begin{figure}[\floatplace]
\begin{center}
\includegraphics[width=0.5\textwidth]{RPA6322withHsynth}
\end{center}
\caption{\label{fig.RPA6322withHsynth} Top: Signal from RPA-A for a 12.7~km/s impact of a $1.9\times10^{-17}$~g particle on positively charged tungsten.  This signal was among the sharpest and was selected as the model for the matched filter impulse response.  Bottom: Synthetic impulse response constructed by fitting a first-order decaying exponential to the above RPA signal.}
\end{figure}

\begin{figure}[\floatplace]
\begin{center}
\includegraphics[width=0.5\textwidth]{RPAQ6614}
\end{center}
\caption{\label{fig.RPAQ6614} Top: Signal from RPA-A for a 3.9~km/s impact of a $1.8\times10^{-15}$~g particle on positively charged tungsten (in blue).  The output of the matched filter (in red) is scaled by a factor of 1/200 to fit on the axes.  Bottom: Signal from the inductive pickup loop detecting passage of the charged particle.  The green and black dashed lines indicate the time to traverse 20~cm and the pulse height, respectively.  Note that the two time scales are different, but synchronized.}
\end{figure}

\begin{figure}[\floatplace]
\begin{center}
\includegraphics[width=0.5\textwidth]{RPAQ8634}
\end{center}
\caption{\label{fig.RPAQ8634} Signal from RPA-A for an 8.7~km/s impact of a $1.5\times10^{-16}$~g particle on positively charged aluminum (in blue).}
\end{figure}
\clearpage

\subsection{Signal Incidence Dependence on Target}
The number of signal peaks detected using the matched filter signal was quantified with respect to target material and bias and is shown in figure \ref{fig.SignalIncidence}.  While these data do not account for signal strength or any dependence on particle mass and speed, it shows a significant drop in signal incidence rate on the two uncharged aluminum targets compared to the uncharged tungsten.  In addition, there is a significant difference in the incidence rate between positively and negatively charged tungsten.

\begin{figure}[\floatplace]
\begin{center}
\includegraphics[width=0.5\textwidth]{SignalIncidence}
\end{center}
\caption{\label{fig.SignalIncidence} Percentage of events where RPA signal peaks are detected, out of the set of events with computed particle masses and speeds.  Error bars are based on a normal approximation to a binomial process.}
\end{figure}

\subsection{Charge Production Dependence on Impact Parameters}
For peaks identified using the matched filter, the height and integrated peak area of the raw RPA signals were found, as illustrated in figure \ref{fig.RawSignalMetrics6976}.  From the peak height and the transimpedance gain of the RPA, we can compute the maximum current deposited on the RPA.  From the integrated peak area and the transimpedance gain, we can compute the total charge deposited on the RPA.  Scatter plots of the peak current and measured charge as functions of particle mass and speed are shown in figure \ref{fig.RawSignalScatter} for impacts on the unbiased tungsten target.  For tungsten in each of the three bias states (float, +1~kV, -1~kV), the peak current and measured charge were fitted to a power law of the form $y = Cm^\alpha v^\beta$.  In this equation, $y$ is the signal strength (current or charge), $m$ and $v$ are the particle mass and speed, and $C$, $\alpha$, and $\beta$ are the fit parameters.  The aluminum and aluminum foil data did not include enough points for a confident fit.  The exponential parameters for tungsten are summarized in table \ref{tab.MVExponents}.

\begin{figure}[\floatplace]
\begin{center}
\includegraphics[width=0.5\textwidth]{RawSignalMetrics6976}
\end{center}
\caption{\label{fig.RawSignalMetrics6976} RPA signal with integrated peak area shaded and peak height indicated in red.}
\end{figure}

\begin{figure}[\floatplace]
\begin{center}
\includegraphics[width=0.45\textwidth]{RawPeakHeightScatter-Current}
\includegraphics[width=0.45\textwidth]{RawPeakAreaScatter-Charge}
\end{center}
\caption{\label{fig.RawSignalScatter} Peak current and measured charge deposited on RPA-A as a function of mass and speed for impacts on unbiased tungsten.}
\end{figure}

\begin{table}[\floatplace]
\begin{center}
\begin{tabular}{c|cccc}
&Metric&$\alpha$&$\beta$&$\beta/\alpha$ \\
\hline
\multirow{2}{*}{+1~kV}&Height (current)&	0.21&	0.68&	3.28\\
&Area (charge)&	0.33&	0.80&	2.43\\
\hline
\multirow{2}{*}{Float}&Height (current)&	0.08&	0.40&	4.77\\
&Area (charge)&	0.28&	1.17&	4.20\\
\hline
\multirow{2}{*}{-1~kV}&Height (current)&	0.12&	1.11&	9.36\\
&Area (charge)&	0.40&	1.90&	4.72
\end{tabular}
\end{center}
\caption{\label{tab.MVExponents} Exponential parameters for the power law fit to signal strength using impact signals on the tungsten target.}
\end{table}

For uncharged tungsten, we find that the measured charge follows the relationship
\begin{equation}
\label{eq.QProd}
1.57 \times 10^{-15} m^{0.5}v^2,
\end{equation}
where $m$ is the projectile mass in g and $v$ is the impact speed in m/s.  The constant multiplier is only for the measured charge in the RPA, and would be a factor of 5 to 35 higher for total charge production, depending on the cone angle of the plasma plume.  The exponential parameters are lower than those reported by \citet{McBride1999}, whose relationship for charge production is
\begin{equation}
\label{eq.McBride}
2.23 \times 10^{-14} m^{1.02}v^{3.48},
\end{equation}
with the same units as Equation \ref{eq.QProd}.

\subsection{Plasma Expansion}
In order to compute the plasma expansion speed, the temporal difference was measured between the matched signal peaks in the two RPAs, as depicted in figure \ref{fig.DualRPA7021}.  Assuming that the plasma plume geometry is insensitive to the angular difference between the two RPAs, we compute the plasma expansion speed using the baseline range difference of 75~mm.  A scatter plot and histogram of expansion speeds resulting from impacts on uncharged tungsten are shown in figure \ref{fig.PlasmaExp}.  There were not enough data points from the other target configurations to yield a meaningful histogram, but the computed expansion speeds were similar in magnitude to those from uncharged tungsten.  Based on these data, there is no clear dependence of expansion speed on projectile mass or speed.  However, this result could be an artifact of the Van de Graaff accelerator's 2~MV operating curve.

\begin{figure}[\floatplace]
\begin{center}
\includegraphics[width=0.5\textwidth]{DualRPA7021}
\end{center}
\caption{\label{fig.DualRPA7021} Signal from both RPAs for a 4.7~km/s impact of a $1.86\times10^{-12}$~g particle on unbiased tungsten (in blue).  The output of the matched filter (in red) is scaled by a factor of 1/200 to fit on the axes.  The dashed green lines indicate the peaks in each signal, yielding a plasma expansion speed of 17.2~km/s for this event.}
\end{figure}

\begin{figure}[\floatplace]
\begin{center}
\includegraphics[width=0.45\textwidth]{PlasmaExpScatter}
\includegraphics[width=0.45\textwidth]{PlasmaExpHist}
\end{center}
\caption{\label{fig.PlasmaExp} Plasma expansion speed for impacts on uncharged tungsten.  The speed is computed from the time difference between signal peaks in the two RPAs.}
\end{figure}

\section{Conclusion}
\label{sec.Conclusion}

The results presented in this paper describe a signal strength dependence on both target material and bias.  The aluminum target yielded weaker signals than the tungsten, perhaps due to its lower density and a correspondingly weaker impact shock.  The charged targets yielded stronger signals than the floated targets, likely because of the accelerating potential and separation of species.  These dependences will be studied further in future testing, and could inform future spacecraft designs in order to mitigate the effect of meteoroid impacts on electrical systems.

The measured charge production has lower mass and velocity exponents compared to previously reported results but the ratio of the exponents is similar.  The difference in total charge production may be attributed to the experimental setup, since the charge detection was performed differently.

Plasma expansion speeds of 10~km/s to 30~km/s are comparable to predicted speeds based on an isothermal plasma model.

Further study is planned at MPIK as well as at a light gas gun facility.  Future tests are expected to better characterize the expanding plasma plume geometry, which will yield an improved measurement of plasma expansion speed.  In the longer-term we plan to measure in situ meteoroid impact events using a small satellite platform.

\section{Experimental objectives}

\section{Van de Graaff accelerator tests}
\subsection{Experimental method}
\subsection{Hardware development}
\subsection{Experimental results}

\section{Light-gas gun tests}
\subsection{Experimental method}
\subsection{Hardware development}
\subsection{Experimental results}

\section{Summary of ground-based experiments}

\section{Intro to thesis body}
\label{sec.IntroToThesisBody}
Blargh!
 \ref{ch.GroundBasedExperiments}
 
 \ref{sec.IntroToThesisBody}
\begin{figure}[htb]
\caption{Global and local coordinate frames}
\label{fig.coords}
\begin{center}
%\scalebox{0.25}{\includegraphics{coords.png}}
\end{center}
\end{figure}
\citep{zubrin1999}\\
\citep{mcinnes2003a}\\
\citep{zubrin1999,mcinnes2003a}\\
\citet{zubrin1999}\\
\citet{mcinnes2003a}\\
\citet{zubrin1999,mcinnes2003a}
% !TEX root = ../Thesis.tex

\chapter{Determining Motion Between Loop Closures}
\label{ch.IcebergGeometry}

This chapter describes how the vehicle trajectory is estimated between loop closure events (Figure \ref{fig:MultiPath}). It starts with an overview of the method developed by Kimball in \cite{Kimball2011b}, which estimated the iceberg's trajectory through the world explicitly. It then shows how solving the same set of equations in the iceberg frame of reference simplifies the estimation, yielding speed increases of several orders of magnitude without sacrificing accuracy.

Terms will be defined as the equations are developed, with an overview of the notation and definitions of commonly-used terms also given in Appendix~\ref{ch.Notation}.

%\section{Geometric Problem Definition}

%This section develops the equations by which DVL data is incorporated into the optimization to enforce motion consistency in the mapping optimization. First, an overview of the previous method is presented. This illustrates the general approach to how DVL information is incorporated, and shows how the changes presented in this thesis provide gains in terms of speed and reduced complexity. For notation conventions, the reader should refer to the Appendix.

%Chapter \ref{ch.LoopClosure} described the process of loop closure detection. This process aligned clouds of multibeam sonar data to provide a rotational and translational offset between two poses that is used to eliminate navigational drift. If the navigation estimate were perfect, this offset error would be zero. However, in general this is not the case, and the motion estimate must be altered in order to satisfy the loop closure constraint. The challenge is to distribute these errors in such a way as to minimize the resulting odometry errors, while still satisfying loop closure. In broad terms, loop closure gives a starting and ending point to the vehicle's trajectory, and the optimizer must choose the most likely path to connect these points out of all possible paths, using whatever sensor observations it has available to it, as shown in Figure \ref{fig:MultiPath}. This is step 2a of the iceberg SLAM diagram originally presented in Chapter \ref{ch.ProblemStatement}, and shown again for convenience in Figure \ref{fig:icebergSLAMmotion}


%\begin{figure}[htb]
%   \centering
%   \includegraphics[scale=.6]{../graphics/IcebergSLAM/IcebergSLAMphotos/Slide4.png} % requires the graphicx package
%   \caption{Motion estimation in the iceberg SLAM problem}
%   \label{fig:icebergSLAMmotion}
%\end{figure}

%This chapter presents improvements on the method for estimating the vehicle's iceberg-relative trajectory between loop closure events using available sensors, developed in \cite{Kimball2011b}. Generating an accurate, self-consistent iceberg map  At a high level, the method revolves around accurate determination of the vehicle's pose \emph{within the frame of reference of the iceberg,} as described in Chapter \ref{ch.ProblemStatement}. The method in \cite{Kimball2011b} assumed that the vehicle's pose in the \emph{inertial} frame could be determined, and estimated the motion of the iceberg based on the measurements. The vector difference between these two quantities is the pose in the iceberg frame. The method presented here estimates the vehicle pose in the iceberg frame directly. In doing so, it reduces the number of estimation parameters by several orders of magnitude, yielding corresponding speed increases. In doing so, it also puts the problem into a form that facilitates the use of pre-existing, highly-optimized solvers.

%The method uses the same basic geometric relations as a starting point, and improves on them in two key regards. First, several mathematical substitutions are made to reduce the number of parameters to be estimated, and eliminate inertial terms from the optimization. Second, all derivatives are taken in the moving reference frame. These changes yield a smaller estimation problem with no sacrifice in accuracy, and remove the need for the vehicle to be supplied with inertial navigation fixes, which is an operational benefit.

%The optimization framework presented in \cite{Kimball2011b} had the structure shown in Figure \ref{fig:NestedLoop}. The optimization is split into linear and nonlinear parts. The inner loop estimation vector $\hat{\bar{x}}_{ls}$ comprises $O(10)$ iceberg motion spline control points $P_x$ and $O(100)$ - $O(1000)$ iceberg-fixed points, denoted $S_i$ in Figure \ref{fig:LoopEquation}.

%\begin{align}
%\hat{\bar{x}}_{ls} &= 
%        \left[\begin{array}{c}
%        \vec{P_x}\\
%        \vec{S_1}\\
%        \vdots\\
%        \vec{S_n}
%        \end{array}
%        \right]
%\end{align}

%Although the inner loop is a linear operation, the number of parameters to be estimated grows linearly with the map size, and can result in slow performance for large icebergs. The changes presented enable the elimination of the inner loop estimation entirely, resulting in an optimization were only the heading of the iceberg as a function of time need be estimated. This can be seen graphically in Figure \ref{fig:NestedLoop2} The specifics of the manipulations used to eliminate this part of the optimization will be developed fully in the following sections. 

\begin{figure}[h!]
   \centering
   \includegraphics[scale=.4]{../graphics/dvlMultiplePaths.png} % requires the graphicx package
   \caption{The DVL constrains the length of the path taken between loop closure events, but there remain many possible paths that could achieve the required offset.}
   \label{fig:MultiPath}
\end{figure}

\section{Previous method}
The method developed in \cite{Kimball2011b} assumed that the vehicle's inertial trajectory could be measured or estimated independent of the profiling optimization, and attempted to estimate the iceberg's motion during the course of the mapping run. This estimate could then be subtracted from the inertial trajectory, yielding the iceberg-relative trajectory. 

The starting point for the estimation problem is Equation \ref{eqn:BasicLoop}, the ``loop equation," illustrated in Figure \ref{fig:LoopEquation}. The optimizer seeks to minimize alignment error while ensuring that the terms of this equation remain consistent with one another. 

\begin{align}
\vec{r}^{~\nicefrac{B_{cm}}{O}} + \vec{r}^{~\nicefrac{S_i}{B_{cm}}} &= ~\vec{r}^{~\nicefrac{V_{cm}}{O}} + \vec{r}^{~\nicefrac{S_i}{V_{cm}}}
\label{eqn:BasicLoop}
\end{align}

\begin{figure}[htbp]
   \centering
   \includegraphics[width=.8\textwidth]{../graphics/IcebergGeometry.png} % requires the graphicx package
   \caption{The geometry of iceberg mapping. }  
    \label{fig:LoopEquation}
\end{figure}

\noindent The loop equation gets its name from the fact that the two paths to get from $O$ to $S_i$ are equivalent, and define a ``loop." Here, $O$ is an arbitrary point fixed in inertial space (chosen to be the origin, for convenience) and $S_i$ is the iceberg-fixed point ensonified by the sonar beam. Similarly, $B_{cm}$ and $V_{cm}$ are the iceberg and vehicle centers of mass, respectively. The center of mass is a common choice for location of the origin of a body-fixed coordinate system, so it is used here, though any iceberg, and vehicle-fixed points could have been used.

In \cite{Kimball2011b}, this equation is differentiated in the inertial frame, $I$

\begin{equation} % requires amsmath; align for no eq. number
   \frac{~^Id}{~dt}(\vec{r}^{~\nicefrac{V_{cm}}{O}} + \vec{r}^{~\nicefrac{S_i}{V_{cm}}} = ~ \vec{r}^{~\nicefrac{B_{cm}}{O}} + \vec{r}^{~\nicefrac{S_i}{B_{cm}}})\\
   \end{equation}
   
\noindent which after simplification yields 
   
       \begin{equation}
    ~^{I}\vec{v}^{V_{cm}} ~+~ \frac{~^Vd}{~dt}(\vec{r}^{~\nicefrac{S_i}{V_{cm}}} ) ~+~ ^{I}\vec{\omega}^{V_{cm}}\times\vec{r}^{~\nicefrac{S_i}{V_{cm}}}  = 
    ~^{I}\vec{v}^{B_{cm}} ~+~^{I}\vec{\omega}^{B}\times\vec{r}^{~\nicefrac{S_i}{B_{cm}}}. \\
\end{equation}

By the definition of $S_i$ as the iceberg-fixed point ensonified by the DVL,  $\vec{r}^{~\nicefrac{S_i}{V_{cm}}}$ can be written 
\begin{equation}
\vec{r}^{~\nicefrac{S_i}{V_{cm}}}= r_{i,t}\hat{d}_{i}  
\label{eqn:SiDef}
\end{equation}

\noindent where $\hat{d}_{i} $ is the fixed (in vehicle frame) unit vector of the $i_{th}$ DVL beam and $r_{i,t} $ is the corresponding range measurement at time $t$.  

Taking the dot product of this equation with $\hat{d}_{i} $ yields,

    \begin{align}
    \hat{d}_{i} \cdot~^{I}\vec{v}^{V_{cm}} ~+~ \frac{~dr_{i,t}}{~dt}  ~+~ \underbrace{\hat{d}_{i} \cdot^{I}\vec{\omega}^{V}\times{~r_{i,t}\hat{d}_{i}}}_{0}  &=
     \label{eqn:expensive}
    \hat{d}_{i} \cdot~^{I}\vec{v}^{B_{cm}} ~ + ~\hat{d}_{i} \cdot\left(^{I}\vec{\omega}^{B}\times\vec{r}^{~\nicefrac{S_i}{B_{cm}}}\right)  \\
    \underbrace{\hat{d}_{i} \cdot~^{I}\vec{v}^{V_{cm}}}_{\text{Assumed known}} ~+~ \underbrace{\frac{~dr_{i,t}}{~dt} }_{\text{From DVL}}  &=
    \underbrace{\hat{d}_{i} \cdot~^{I}\vec{v}^{B_{cm}}}_{\text{Estimated}} ~ + ~\hat{d}_{i} \cdot\left(\underbrace{^{I}\vec{\omega}^{B}}_{\text{From outer loop}}\times\underbrace{\vec{r}^{~\nicefrac{S_i}{B_{cm}}}}_{\text{Estimated}}\right)  
         \label{eqn:expensive2}
\end{align}

The previous method arranged the terms of Equation \ref{eqn:expensive} within the nested loop optimization shown in Figure \ref{fig:NestedLoop}. The terms that are assumed known or are estimated in the outer loop make up the rows of the matrix $A$. The $\frac{~dr_{i,t}}{~dt}$ term is measured directly by the DVL, and corresponds to entries of the vector $b$. The remaining parameters to be estimated are concatenated in the vector $x_{ls}$ and solved via linear least squares in the inner loop optimization. 

\begin{equation}
    \hat{\bar{x}}_{ls} = A^{\dagger} b
    \label{eqn:LeastSquares}
\end{equation}

\noindent where $\hat{\bar{x}}_{ls}  \in \mathbb{R}^{ 2P_x + 2N_{s_i}}$. $P_x$ is the number of spline control points used to estimate the iceberg's translation and inertial velocity, and $N_{S_i}$ is the number of iceberg-fixed ``map points," $S_i$.


Note that the development here differs slightly from \cite{Kimball2011b} in that the four DVL beams are not abstracted to a single point before optimizing. However, the fundamental concept and mechanics of the solution method remain the same.

\begin{figure}[htbp]
   \centering
   \includegraphics[scale=.4]{../graphics/KimballIterativeOpt.png} % requires the graphicx package
   \caption{Iterative optimization structure employed in \cite{Kimball2011b}. The nonlinear heading parameters $\psi$ are chosen in the outer loop and fed to the inner loop as constants. $\hat{\bar{x}}_{ls}$ is a vector of O(10) motion parameters and O(1000) iceberg-fixed points to be estimated at each optimization step.  \emph{Graphic courtesy of Peter Kimball}.}
   \label{fig:NestedLoop}
\end{figure}




\section{Improvements to the method}

Two modifications are introduced to the method presented in \cite{Kimball2011b}.  The first simplifies the problem by representing the iceberg-fixed points as the difference between other known or already estimated variables. This uses all the same information but reduces the number of estimation parameters, yielding large speed increases. The second modification recasts the problem entirely in the iceberg frame of reference. In doing so, it measures the vehicle's iceberg-relative trajectory directly. It yields similar speed increases, and also eliminates the need to have an estimate of the vehicle's inertial position, which can be advantageous from an operations standpoint.

\begin{figure}[htbp]
   \centering
   \includegraphics[scale=.4]{../graphics/HammondIterativeOpt.png} % requires the graphicx package
   \caption{Modification of the optimization framework. The inner loop optimization has been replaced by a deterministic loop closure evaluation based on inertial navigation and DVL. The resulting optimization can be run several orders of magnitude faster than the successive loop method.}
   \label{fig:NestedLoop2}
\end{figure}

\subsection{Reducing the number of estimation parameters}

%One problem with the method given in Equations \ref{eqn:expensive} and \ref{eqn:LeastSquares} is that the number of parameters $\vec{r}^{~\nicefrac{S_i}{B_{cm}}}$ grows with the size of the iceberg, and can make the inner loop optimization very large. 

The number of estimation parameters required in the inner loop can be reduced by several orders of magnitude without loss of accuracy. This is done by recognizing that the unknown quantity $\vec{r}^{~\nicefrac{S_i}{B_{cm}}}$ can be represented as the difference between other quantities that are either assumed to be known, or that are already estimated. The iceberg-fixed sounding location is the vector difference of the sonar beam head's inertial position estimate and the iceberg center of mass position estimate: 

\begin{equation}
    \label{eqn:SiSubstitution}
    \vec{r}^{~\nicefrac{S_i}{B_{cm}}}  = ~\vec{r}^{~\nicefrac{V_{cm}}{O}} + \vec{r}^{~\nicefrac{S_i}{V_{cm}}} - \vec{r}^{~\nicefrac{B_{cm}}{O}}
\end{equation}

Substituting \ref{eqn:SiSubstitution} into \ref{eqn:expensive} yields the new equation

     \begin{equation}
 \hat{d}_i\cdot\underbrace{^{I}\vec{v}^{V_{cm}}}_\text{known} ~-
  \hat{d}_i\cdot(\underbrace{^{I}\vec{\omega}^{B}}_{\text{outer loop}}\times~\underbrace{\vec{r}^{~\nicefrac{V_{cm}}{O}}}_\text{known}) ~+
 \underbrace{ \frac{~dr_{i,t}}{~dt}}_{\text{from DVL}}  
  =~\hat{d}_i\cdot(~^{I}\vec{v}^{B_{cm}} ~-
 ~^{I}\vec{\omega}^{B}\times\vec{r}^{~\nicefrac{B_{cm}}{O}}).
 \label{eqn:getIt}
\end{equation}

The terms of Equation \ref{eqn:getIt} can again be arranged such that all terms that are either assumed known, measured, or estimated in the outer loop are on the left hand side, and all the terms estimated in the inner loop are on the right hand side. Noting the basis in which each vector is written, and writing out all necessary rotation matrices,

\begin{equation}
 \underbrace{\hat{d}_i^{\intercal}~^{V}R^I ~(^{I}\vec{v}^{V_{cm}} ~-~[^{I}\vec{\omega}^{B}]_\times\vec{r}^{~\nicefrac{V_{cm}}{O}}) ~+
 \frac{~dr_{i,t}}{~dt} }_{``\text{Measurement}" ~i ~\text{at time}~ j}
  =~\underbrace{\hat{d}_i^{\intercal}~{^{V}R^I} ~((\dot{B}_x(t_j)~-
 [^{I}\vec{\omega}^{B}]_\times B_x(t_j))}_\text{one row of the measurement matrix}\vec{P}_x.
 \label{eqn:finaldvl}
\end{equation}

\noindent where $[~~~]_\times$ is the matrix representation of the vector cross product, $B_x(t_j)$ is the $2\times2N_\text{control points}$ matrix of position spline basis functions evaluated at time $t_j$, and $\dot{B}_x(t_j)$ is the equivalent matrix for the basis derivatives. $\vec{P}_x$ are the iceberg position spline control points being estimated in the inner optimization loop.

The important result is the estimation vector $x_{ls}$ now contains \emph{only} the points $P_x$. The ``map points" that caused the vector in Equation \ref{eqn:LeastSquares} to have high dimensionality are absent. This improves greatly the speed of the inner loop optimization, as the number of parameters to estimate is on the order of 10 instead of hundreds or thousands.

This substitution can be used as a replacement to the previous method, with the values of $A$ and $b$ in Equation \ref{eqn:LeastSquares} changed only slightly. Since least squares computational complexity is proportional to the square of the number of parameters estimated, this reduction yields significant savings. However, at this point, the method still requires knowledge of the inertial position and velocity of the vehicle, which can be difficult to determine accurately without external aids. The next section performs the same calculations as above, but defined entirely with respect to the moving iceberg frame, and with all derivatives taken in that frame, to eliminate this need. 

%TO DO: I THINK I NEED TO BACK UP THE STATEMENT THAT THESE POINTS CAN BE ELIMINATED WITHOUT SACRIFICING ACCURACY, SINCE THEY'RE ONLY OBSERVED ONCE (EXCEPT FOR LOOP CLOSURE POINTS), AND THE COSINE-DEPENDENT RANGE-VELOCITY AMBIGUITY.

\subsection{Solution in the iceberg reference frame}
\label{sec.MainContrib}

To remove inertial terms from the optimization, a similar substitution is made to the one shown in Equation \ref{eqn:SiSubstitution}. Equation \ref{eqn:BasicLoop} can be rewritten as

\begin{figure}[htbp]
   \centering
   \includegraphics[width=.8\textwidth]{../graphics/IcebergGeometryRelative.png} % requires the graphicx package
   \caption{The inertial positions of the iceberg and vehicle, shown in Figure \ref{fig:LoopEquation}, can be replaced with their difference, $ \vec{r}^{~\nicefrac{V_{cm}}{B_{cm}}}$. This is the vehicle's iceberg-relative position, which is the parameter needed to estimate accurately in order to create a map. }
    \label{fig:LoopEquationBergFrame}
\end{figure}

\begin{align}
\vec{r}^{~\nicefrac{B_{cm}}{O}}  - ~\vec{r}^{~\nicefrac{V_{cm}}{O}} + \vec{r}^{~\nicefrac{S_i}{B_{cm}}} &= \vec{r}^{~\nicefrac{S_i}{V_{cm}}}
\label{eqn:Loop2}
\end{align}
and the difference of inertial terms can be expressed as 

\begin{align}
\vec{r}^{~\nicefrac{B_{cm}}{O}}  - ~\vec{r}^{~\nicefrac{V_{cm}}{O}} + \vec{r}^{~\nicefrac{S_i}{B_{cm}}} &= \vec{r}^{~\nicefrac{V_{cm}}{B_{cm}}}
\label{eqn:InertialSubstitution}
\end{align}

\noindent i.e. the position of the vehicle in the frame of reference of the icberg. Substituting \ref{eqn:InertialSubstitution} into \ref{eqn:Loop2} yields the simplified loop equation, written only in iceberg-relative terms:

\begin{align}
\vec{r}^{~\nicefrac{V_{cm}}{B_{cm}}}+ \vec{r}^{~\nicefrac{S_i}{B_{cm}}} &= \vec{r}^{~\nicefrac{S_i}{V_{cm}}}
\label{eqn:IcebergLoop}
\end{align}

Removing the inertial terms allows straightforward differentiation in the iceberg frame of reference


\begin{align} % requires amsmath; align for no eq. number
   \frac{~^Bd}{~dt}\left(\vec{r}^{~\nicefrac{V_{cm}}{B_{cm}}}+ \vec{r}^{~\nicefrac{S_i}{B_{cm}}}\right.& =\left. \vec{r}^{~\nicefrac{S_i}{V_{cm}}}\right)\\
   ~^B\vec{v}^{V_{cm}}  + \frac{~^Bd}{~dt}\left(\vec{r}^{~\nicefrac{S_{i}}{V_{cm}}}\right) &= \underbrace{\frac{~^Bd}{~dt}\left(\vec{r}^{~\nicefrac{S_i}{V_{cm}}}\right)}_{\vec{0}} \\
   ~^B\vec{v}^{V_{cm}}  + \frac{~^Bd}{~dt}\left(\vec{r}^{~\nicefrac{S_{i}}{V_{cm}}}\right) &= \vec{0}
   \label{eqn:dLoopdB}
\end{align}

Substituting Equation \ref{eqn:SiDef} into \ref{eqn:dLoopdB} and expanding yields

\begin{align} % requires amsmath; align for no eq. number
   ~^B\vec{v}^{V_{cm}}  + \frac{dr_{i,t}}{dt}\hat{d}_i + ~^B\vec{\omega}^V \times r_{i,t}\hat{d}_i&= \vec{0}
   \label{eqn:gettingClose}
\end{align}
and then taking the component  in the direction of $\hat{d}_i$ via the inner product

\begin{equation} % requires amsmath; align for no eq. number
   \hat{d}_i \cdot ~^B\vec{v}^{V_{cm}}  + \frac{dr_{i,t}}{dt} + \underbrace{ \hat{d}_i \cdot \left( ~^B\vec{\omega}^V \times r_{i,t}\hat{d}_i\right)}_{0} = 0
   \label{eqn:bodyVelocity}
   \end{equation}
   \begin{equation}
   \hat{d}_i \cdot ~^B\vec{v}^{V_{cm}} = - \frac{dr_{i,t}}{dt}
   \label{eqn:boom}
\end{equation}

Equation \ref{eqn:boom} allows the body components of the vehicle's velocity in the iceberg's reference frame to be measured to high precision via least squares. This is exactly the normal principle of operation for a DVL in ``bottom track." 

   \begin{equation}
   ~^B\vec{v}^{~V_{cm}}_V = A_{DVL}^\dagger b
   \label{eqn:LSDVL}
\end{equation}

\noindent where each row of $A_{DVL}$ comprises the body-fixed unit vector of one of the four DVL beams $\left[\hat{d}_i\right]_V^\intercal$, and each entry in $b$ is the corresponding Doppler measurement $ -\frac{dr_{i,t}}{dt} $. However, the vehicle's heading in the iceberg cannot be observed directly by the DVL, so the the rotation matrix $~^BR^{~V}(t) $ must be estimated in the outer loop. Given this transformation between the vehicle frame and iceberg frame, 

   \begin{equation}
   ^B\vec{v}^{~V_{cm}}_B= ~^BR^{~V}~ \cdot~^B\vec{v}^{~V_{cm}}_V
   \label{eqn:boomLinAlg}
\end{equation}

\noindent the quantity $^B\vec{v}^{~V_{cm}}_B$ can be integrated to obtain the vehicle's iceberg-relative position. This is how the large linear least-squares inner loop optimization shown in Figure \ref{fig:NestedLoop} can be replaced by a deterministic trajectory propagation, as shown in Figure \ref{fig:NestedLoop2}. This also makes the problem's structure essentially identical to mapping tasks commonly solved via SLAM in the terrestrial robotics community, a link that will be developed fully in Chapter \ref{ch.GraphSLAM}.

\section{Heading ambiguity}

Up until this point, it has been assumed that the heading of the vehicle in the frame of reference of the iceberg could be determined by estimation in the outer loop of the optimization, but no explanation of how this could be accomplished has been given. This section gives a short discussion of the observability of this parameter, practical considerations for its estimation, and how the problem changes if a precise high-overlap multibeam sonar is included in the sensor suite.

\subsection{Heading observability}

Neither the multibeam mapping sonar, nor the DVL can determine the vehicle's instantaneous iceberg-relative heading rate of change. Together, they provide in theory a small amount of heading observability, but in practice, the signal to noise ratio is very low.

The mapping multibeam sonar cannot observe heading changes because it is oriented perpendicular to the direction of travel, as shown in Figure \ref{fig:beamOrientation}. This provides the widest coverage for mapping purposes, but means that there is generally no overlap between successive scans. Without overlap, no relative heading information can be obtained. 

\begin{figure}[h]
   \centering
   \includegraphics[width=.8\textwidth]{../graphics/ScanMatch3.png} % requires the graphicx package
   \caption{Simulated AUV scans taken at two times, roughly 20 meters apart. The multibeam mapping sonar sweeps out a wide area when oriented perpendicular to the direction of travel. This orientation does not allow scan matching for odometry, since the beam pattern does not overlap from scan to scan.}
   \label{fig:beamOrientation}
\end{figure}

The DVL can not provide heading information unless strong assumptions are made about the terrain. For instance, if terrain is assumed to be planar, then a plane can be fit to the beams, and heading changes can be estimated by finite differencing. However, for arbitrary geometries, this approach will not yield good results. Because the DVL measures beamwise components of velocity, rotations are unobservable by Doppler shift. 

When used together, the mapping multibeam and DVL theoretically can provide a small amount of heading observability. This is shown in Figure \ref{fig:headingObservability}. The signal to noise ratio of this observation is low enough that practical implementation would pose a challenge, but it is presented here for completeness. The key concept is that the forward angle of the DVL and the high linear density of the multibeam soundings  enable multiple ensonifications of the same patch of ice. As shown in Figure \ref{fig:headingObservability}, the DVL (green beam) ensonifies a patch of ice at $t1$. Later, at $t2$ the vehicle's multibeam sweeps over the same patch of ice. The position at time $t2$, $x_2$ is given by the equation

\begin{align}
x_2 = x_1 + \int_{t_1}^{t_2}~^BR^V(t)~u_{\tiny{DVL}}dt
\end{align}

The loop equation that can be used to generate an error signal is then given by

\begin{align}
\int_{t_1}^{t_2}~^BR^V(t)~u_{\tiny{DVL}}dt = ^BR^V(t_1)d_1 -  ^BR^V(t_2)m_2
\end{align}

\noindent where $d_i$ and $m_i$ are the body-fixed DVL and multibeam measurements, respectively, at time $t_i$.

\begin{figure}[h]
   \centering
   \includegraphics[width=.8\textwidth]{../graphics/headingObservability.png} % requires the graphicx package
   \caption{DVL and Multibeam used together can, in theory, provide information on heading.}
   \label{fig:headingObservability}
\end{figure}

Any odometry error that accumulates between $t_1$ and $t_2$ will show up as a difference between the expected range to the patch and the actual range. This is shown by the ghosted AUV at $t_2$. Navigation error due to the iceberg's rotation has caused its position estimate to pull away from the iceberg relative to the true location (solid AUV). As a result, an error, shown at bottom, develops between the DVL reading at the first time, and the multibeam at the second time. This difference can provide information on the accumulated heading error.

There are two practical challenges to employing this approach. The first is essentially a signal-to-noise ratio problem from trying to measure extremely small quantities with noisy sensors. The second again has to do with ambiguity in the data correspondence problem. 

For the first challenge, assume that the data correspondence can be solved. For example, at 40m standoff distance and a forward DVL angle of $30^\circ$ from the wall, $m_2$ would coincide with $d_1$ after 23m of travel, or 15s at 1.5 m/s. If the iceberg is rotating at 20 deg/hr, which is fast for an iceberg, this would introduce about 1 mm of of translational error. Assuming that both DVL and multibeam have ranging errors of 1\% of distance traveled, the problem can be stated simply: determining short-term heading drift by comparing DVL and multibeam would require detecting discrepancies of a fraction of a millimeter with a sensor that returns $40m \pm 40 cm$.

There remains the ambiguity in data association between the DVL and multibeam returns. In reality each sonar return is a spatial and temporal average over some finite area and span. While this makes it more likely that DVL and multibeam returns have some overlap, this averaging introduces additional ambiguity into the measurement. While it may be possible to perform some iterative process to estimate heading similar to ICP, where correspondences are made and broken based on current estimate of position at each iteration, such a process would be very slow, and the potential gains dubious. So, while in theory it may be possible to observe the iceberg's heading rate directly, in practice, it is easier and more accurate to rely on some reasonable assumptions about the iceberg's motion, such as small, slowly-varying rates, and treat the motion as a bias to the vehicle's accurate inertial heading sensor.


%\begin{figure}[h]
%   \centering
%   \includegraphics[width=.8\textwidth]{../graphics/ScanMatch3a.png} % requires the graphicx package
%   \caption{Top view of multibeam sonar at two times to highlight the lack of overlap between scans. Without overlap, no odometry information can be obtained.}
%   \label{fig:NoMatch}
%\end{figure}

%\begin{figure}[htbp]
%   \centering
%   \includegraphics[width=.8\textwidth]{../graphics/ScanMatch4a.png} % requires the graphicx package
%   \caption{Top view of the iceberg mapping problem. If the sensor field of view has significant extent in the direction of travel, scan matching can provide odometry}
%   \label{fig:ScanMatch1}
%   \end{figure}
%    \begin{figure}[h]
%   \centering
%   \includegraphics[width=.8\textwidth]{../graphics/ScanMatch4b.png} % requires the graphicx package
%   \caption{Odometry corrected by scan matching. This readily extends to the full 3d case using techniques like Iterative Closest Point.}
%   \label{fig:ScanMatch2}
%\end{figure}



\subsection{Modeling iceberg heading as a heading rate drift}

Making a reasonable assumption about the rotational motion of the iceberg allows its motion to be modeled as a time-varying heading rate bias. The key assumption here is that the iceberg's large inertia prevents it from changing its angular rate very quickly. 

There are infinitely many paths of a fixed length that can join two points. These smooth iceberg rotation assumptions allow the use of the AUV's accurate compass as a starting point to find the most likely trajectory. Additionally, as more loop closure events are added, the number of highly probable trajectories becomes fewer and fewer.

Chapter \ref{ch.Results} makes this smoothness assumption and models the iceberg's rotational motion as a slowly-varying bias on the vehicle's heading rate. The details of how this information is incorporated into the solution are presented in Chapter \ref{ch.GraphSLAM}.

\subsection{Extracting iceberg-relative heading from scan matching}

\begin{figure}[h]
   \centering
   \includegraphics[width=.8\textwidth]{../graphics/ScanMatch1.png} % requires the graphicx package
   \caption{Multibeam sonar could be used for odometry if mounted in the horizontal plane as shown. In fact, the vehicle has a low-precision multibeam sonar in this orientation, but it lacks sufficient accuracy to outperform modeling iceberg motion as compass drift.}
   \label{fig:ScanMatchMultibeam}
\end{figure}

In order to perform scan matching for odometry, the sonar would need to be oriented in the horizontal plane, as in Figure \ref{fig:ScanMatchMultibeam}. 

By incrementally aligning these high-overlap scans, a direct measurement can be made of the vehicle's iceberg-relative heading rate. There is no longer a need to make assumptions about the iceberg's motion, other than that it is planar. Additionally, given the vehicle's compass, the iceberg's rotation could be extracted from the measurements.

To generate these measurements, scans are aligned using ICP with 3-DOF point-to-plane distance penalty. The heading offset for each scan match is used to update the vehicle's heading estimate, rather than compass and bias estimate. The difference between the sonar-based heading estimate and the compass heading gives a reasonable approximation of the iceberg's true heading rate. 

\begin{figure}[h]
   \centering
   \includegraphics[width=\textwidth]{../graphics/filtIgnxHeading.png} % requires the graphicx package
   \caption{Comparing the horizontal sonar-derived heading estimate with the compass provides and estimate of the iceberg's rotational motion.}
   \label{fig:filtIgnx}
\end{figure}

Sample results are presented in Figure \ref{fig:filtIgnx}. These results were generated in simulation with no ranging error on the sonar. In practice, range errors would cause performance degradation. With the low-cost obstacle avoidance sonars currently mounted on the vehicle, the accuracy is insufficient to outperform the compass drift model, but work toward improving the signal processing algorithms to make this approach feasible is underway. Adding a high-precision multibeam sonar of the type used for mapping in horizontal orientation would work, but is prohibitively expensive.

\section{Summary}

A number of parameter substitutions simplify the method presented in \cite{Kimball2011b}. These changes reduce the estimation state space by several orders of magnitude, which speeds up computation dramatically. Additionally, performing all analysis in the frame of reference of the iceberg, rather than the inertial frame, eliminates all inertial quantities from the equations. As a result, mapping missions may be run without the need to supply the vehicle with external navigation fixes, which increases vehicle autonomy and is beneficial from an operational complexity standpoint. 

These changes cast the iceberg mapping problem in a form that is common in the SLAM literature, allowing the application of powerful, highly optimized solution methods to further improve the speed and accuracy of iceberg mapping. This will be demonstrated in detail in Chapter \ref{ch.GraphSLAM}, where the iceberg mapping task is developed and solved using the GraphSLAM technique. 


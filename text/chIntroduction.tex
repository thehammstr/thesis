% !TEX root = ../Thesis.tex

\chapter{Introduction}
\label{ch.Introduction}

Iceberg profiling is the creation of 3-D shape reconstructions of the submerged portion, or keel, of free-drifting icebergs. An example of an iceberg profile is shown in Figure \ref{fig:idealMap}. These accurate 3-D shape profiles, reconstructions, or simply ``maps" are useful in a number of scientific and industrial applications.  In their most basic representation, these reconstructions consist of point clouds of range data collected by multibeam sonar. The point clouds can then be converted to other formats depending on the application.

\begin{figure}[!htb]
   \centering
   \includegraphics[width=.8\textwidth]{../graphics/icebergs-shadow-350.jpg} % requires the graphicx package
   \caption{Antarctic tabular iceberg at sunset. Photo courtesy of MBARI. \emph{www.mbari.org}}
   \label{fig:IcebergSunset}
\end{figure}

This thesis extends the current state of the art to enable full-depth iceberg profiling using Autonomous Underwater Vehicles (AUVs) as shown in Figure \ref{fig:AUVdorado1}. Obtaining these profiles with AUVs will yield cost and safety benefits over current methods, and will help expand the state of the art in autonomous field robotics. The method presented is heavily influenced by techniques used to create high-precision bathymetry (depth) maps of static undersea environments using AUVs, as well as Simultaneous Localization and Mapping (SLAM) techniques developed in the terrestrial robotics community to map unknown environments.

\begin{figure}[!htb]
   \centering
   \includegraphics[width=.9\textwidth]{../graphics/ceresResults/images/noDriftNoLinks2.png} % requires the graphicx package
   \caption{Simulated high-resolution iceberg point cloud reconstruction. The magenta contour is the vehicle's inertial trajectory}
   \label{fig:idealMap}
\end{figure}

\begin{figure}[htb]
   \centering
   \includegraphics[width=.8\textwidth]{../graphics/mappingAUV.jpg} % requires the graphicx package
   \caption{A Dorado-class AUV owned by the Monterey Bay Aquarium Research Institute (MBARI).}
   \label{fig:AUVdorado1}
\end{figure}

%They enhance understanding of scientific data collected near the iceberg, providing ``geo-tagged"\footnote{or perhaps more appropriately, ``pago-tagged", from  $\pi \acute{\alpha} \gamma o \sigma$,  the greek word for ``ice"}  measurements. These reconstructions also serve as a navigational tool, providing a means to return to sites of interest for future study using terrain-relative navigation (TRN) techniques. 3-D reconstructions also provide insight into climate science and iceberg melt patterns by allowing accurate volumetric comparison between reconstructions of the same iceberg created at different times. 

%In the energy industry, iceberg profiling is used as a tool to assess risk to infrastructure and other assets like underwater pipelines and oil rigs. Accurate shapes aid in drift trajectory simulations that can predict collisions in time for preventative measures to be taken. 

%They are normally acquired through the use of ship-based sonar or Remotely-Operated Vehicles (ROVs). However, AUVs are increasingly being used for this task, though current methods often make strong assumptions about the iceberg's motion, require the AUV to be constantly fed position updates from a ``tender" ship, or can only map incomplete portions of the iceberg at a time \cite{Kimball2011b}.






%The iceberg's rotation and acceleration make it a non-inertial reference frame, which introduces large apparent navigation errors into the trajectory estimate, causing the projected measurements to be inconsistent. Onboard inertial navigation instruments and infrastructure-based navigation tools like GPS and Ultra-Short Baseline Interferometry (USBL) can only provide accurate navigation information \emph{in the inertial frame.} If the inertial trajectory estimate were used to generate a point cloud, neglecting any iceberg motion, the resulting point cloud would contain warping and inconsistencies. 

%Because of the iceberg's motion, even vehicles with perfect inertial navigation can appear to drift by a large amount relative to the moving iceberg ``terrain." To see the effects of this motion, consider an autonomous underwater vehicle mapping an iceberg as depicted in Figure \ref{fig:introSetup}. The yellow cylinder is the vehicle, and the vertically-oriented fan-shaped pattern represents a multibeam sonar, an important tool for the creation of underwater 3-D reconstructions. The vehicle circumnavigates the iceberg several times while attempting to maintain a constant standoff distance from the ice, sweeping out a swath of sonar data as it travels. If the iceberg were fixed, the vehicle's inertial position could be used to project the multibeam readings into the world frame, creating the self-consistent point cloud shown in Figure \ref{fig:idealMap}. However, if the iceberg is not fixed, then reprojecting the measurements based on the inertial navigation alone will result in the point cloud shown in Figure \ref{fig:Challenge}. This type of inconsistency is very similar to that encountered by robots mapping static environments with low-precision inertial navigation. 


%The strategy employed in this dissertation to correct the warping due to iceberg motion is based on estimating the AUV's position in the iceberg's frame of reference. To do this, the problem is posed within a simultaneous localization and mapping (SLAM) framework. This class of methods uses information within the environment (i.e. the iceberg) to aid in trajectory estimation, which in turn produces a more accurate, self-consistent projection of measurements into the world. Hence the robot simultaneously creates a map and localizes itself within that map. The robot can gain information about its accumulated navigational error by revisiting portions of the terrain it has already traversed and using observed offsets to estimate and correct for those errors. 

%One major challenge with performing SLAM in underwater environments is automatic detection of these so-called ``loop closure events." In shallow, feature-rich environments, such as coral reefs, cameras can be used, and robust image features can allow detection of these events. However, when standoff distances are large, or when there is insufficient texture in the terrain to allow image correlation, cameras provide little aid. In these cases, systems typically rely on accurate inertial guidance to provide a good initial estimate of regions of overlap, and then refine these estimates through point cloud correlation techniques like Iterative Closest Point (ICP). These methods can provide very accurate alignment, but are susceptible to local minima, and generally require good intial estimates of alignment to converge to the correct value. This is problematic when working with free-drifting icebergs, since the environment is a non-inertial reference frame, and even the most accurate inertial navigation system (INS) will not necessarily be able to provide sufficient accuracy for this type of loop closure detection. 





\section{Motivation}

There are two primary motivating missions for the research presented in this dissertation. First, iceberg profiles are useful in a number of scientific and industrial applications. Second, profiling icebergs serves as an analog for space missions, providing useful field experience in robotic exploration and study of extreme environments.

\subsection{Iceberg profiling applications}

Iceberg profiles have several scientific applications that can be broken down by time-scale of interest: immediate, days to weeks, and months to years. 

In the immediate term, having a 3-D map of the iceberg allows location-referencing of any data collected in its vicinity. This provides high resolution spatial context for the data, and can aid in identifying points of interest for further study. For instance, tagging salinity readings with their iceberg-relative location can yield insight on how geometry affects the rate at which icebergs deposit minerals like iron into the sea. Iron can act as a fertilizer for phytoplankton, which play a key role in sequestering carbon from the atmosphere.

In the days and weeks after their creation, iceberg reconstructions can be used to improve temporal resolution of data. Having iceberg-referenced data removes spatio-temporal ambiguity in data collected on different surveys, enabling observation of how measured quantities evolve over time. Using the same salinity measurements as an example, the spatially-tagged measurements can show how the local rate of mineral deposits changes over time. Additionally, iceberg reconstructions can be used as maps to aid in return-to-site missions, where specific points of interest identified in an initial survey can be revisited. Aided by the map, the robot can perform terrain-relative navigation (TRN) to return to the site, even if the iceberg has drifted and rotated\cite{Kimball2011b}. This would grant scientists the opportunity to deploy sensor suites tailored to particular applications, potentially giving a much clearer picture of what is happening in important areas.

Finally, on time scales of months to years, performing shape reconstruction on the same iceberg allows the calculation of an iceberg's melt rates and melt patterns.

Iceberg profiles also have industrial applications. Energy companies operating in polar regions routinely hire contractors to do ship-based reconstructions of icebergs that pose a risk to their infrastructure. They use these reconstructions to help predict how the icebergs drift with the wind and ocean currents, and in high-fidelity collision simulations to asses the likelihood of damage in the event of impact with a drilling rig or pipeline \cite{ralph2008iceberg} \cite{fuglem1996iceberg}. Of particular importance in the latter case is knowing the depth and shape of the iceberg keel, as a keel dragging across the seafloor can cause catastrophic damage to pipelines. This is known as ``iceberg scour," and motivates the need for profiling the full depth of the iceberg.

%\subsection{Advantages and Disadvantages of AUVs}
%
%AUVs are an attractive alternative to ship-based methods of iceberg study. Typically, a ship will take an array of measurements at several points around an iceberg. Ships must maintain a large standoff distance when studying icebergs due to calving danger or the presence of brash ice, and may not be able to see the entire depth of the iceberg. Using a tethered ROV can allow closer approaches, but they maneuver slowly, and require the ship to constantly tend them when it might be used for other scientific purposes. Additionally, the tether can become severed or caught on brash ice.
%
%AUVs address all of these concerns. They easily operate under brash ice, can approach closely without putting lives at risk, and do not require a tether. They provide high spatial resolution and can access the entire depth of the iceberg, while performing other scientific measurement, such as temperature, pressure, and salinity readings, or water sampling. 


%There are drawbacks, however: AUVs are limited in the size and power of the sensors they can carry, and can only collect limited volumes of water for later study, when compared to ships. Additionally, they often have limited maneuverability compared with ROVs, which can cause difficulty when operating around complex terrain or ice structure. Acoustic modems allow limited real-time communication with AUVs, but have very low bandwidth, requiring AUVs to be highly self-sufficient, and prohibiting immediate access to the data they collect. Because of this, AUVs will likely augment greatly, rather than completely replace ship-based surveys of icebergs for the foreseeable future. Working together, the surface vessels and underwater vehicles will provide more accurate reconstructions and a much clearer picture of the free-drifting iceberg environment than either method of study could alone.

\subsection{Exploration and study of extreme environments}

Icebergs are harsh environments without navigation infrastructure. In this way, they are similar to the challenging environments encountered during space exploration, like asteroids, comets, or distant planetouds. However, they have the practical advantage of being closer and more accessible. By conducting such real-world robotic exploration and mapping missions, lessons will be learned and techniques developed that will mitigate risks before launching hardware into space. 

\begin{figure}[htb]
   \centering
   \includegraphics[width=.8\textwidth]{../graphics/Europa-moon.jpg} % requires the graphicx package
   \caption{Jupiter's icy moon Europa, a point of great interest for robotic exploration.}
   \label{fig:Europa}
\end{figure}

The research presented here was funded by the NASA Astrobiology Science \& Technology for Exploring Planets (ASTEP) program. A goal of ASTEP is to develop the technology necessary to perform scientific  exploration in extreme environments, such as comets, asteroids, or remote planets and moons like Europa (Figure \ref{fig:Europa}).

The capabilities enabled by this research could one day aid in the study of the origins of life. In particular, a number of preorganic molecular species thought to be necessary for life have been observed in high concentrations in comets and asteroids \cite{Fomenkova1999}. In fact, some theories hypothesize that these biological building blocks were first delivered to Earth via these objects impacting the planet several billion years ago \cite{Chyba1992}. These are some of the reasons why asteroids and comets are scientific points of interest. Several recent missions have studied them, like NASA's Deep Impact and Stardust missions\cite{Ahearn2005,Willcockson1999} and ESA's Rosetta \cite{Bibring2007}.

One challenge in the study of these environments is the lack of navigational infrastructure. There is no equivalent of GPS to supply a robotic vehicle with its position relative to the body or planet, and installing such infrastructure on the surface would likely be prohibitively difficult. Star trackers and inertial guidance systems can be used, but these provide position of the vehicle in inertial space, not relative to the body.

If a detailed shape reconstruction of the body is available, terrain-relative navigation techniques can be used to localize relative to it \cite{Anonsen2006,Anonsen2007}. Additionally, a shape reconstruction is a valuable scientific tool unto itself, allowing other data to be location-referenced and enabling longitudinal studies in the evolution of a body's shape. However, generating a 3-D reconstruction of a body that is moving relative to the inertial reference frame presents several challenges.

\section{Background: previous methods for iceberg profiling}

%A number of methods exist for generating shape estimates of free-drifting icebergs. Aerial observations of icebergs are used to monitor risk to shipping lanes and fixed infrastructure, but can only provide rough size and shape information about the iceberg sail, and virtually no information about the iceberg's keel. Ship-based measurements including tethered ROVs provide high-resolution reconstructions of both the sail and keel, but are labor intensive and expensive. 


\subsection{Aerial Iceberg Profiling}

\begin{figure}[!htb]
   \centering
   \includegraphics[width=.8\textwidth]{../graphics/AerialIcebergs.jpg} % requires the graphicx package
   \caption{ Aerial photographs of icebergs can provide rough information about size and shape. \emph{http://photos.state.gov}}
   \label{fig:IcebergAerial}
\end{figure}

Aircraft are used extensively in tracking and managing icebergs. Coast guards and navies of several nations regularly patrol Arctic and Antarctic waters to detect and track icebergs that might pose a threat to infrastructure or shipping. While this provides an effective means of tracking multiple icebergs' positions over time, it says relatively little about each iceberg's structure. The aircraft are sometimes equipped with LIDAR range scanners \cite{Bunkin2012}, but most often the only measurements available are photographs. From this, rough classification of iceberg type based on sail shape (such as ``drydock," ``pinnacle," or ``blocky" ) is possible. Additionally, keel depth can be estimated heuristically based on maximum waterline length, and decay rate can be inferred from repeated sightings, as shown in \cite{Barker2004}. However, all of these are only first-order approximations of shape; aerial imagery cannot provide detailed iceberg shape profiles of the iceberg's keel. 

\subsection{Ship-based Iceberg Profiling}

\begin{figure}[!htb]
   \centering
   \includegraphics[width=.8\textwidth]{../graphics/FugroIcebergProfiling.png} % requires the graphicx package
   \caption{ A multibeam-equipped profiling ROV is deployed near an iceberg. \emph{www.fugro.com}}
   \label{fig:IcebergROV}
\end{figure}

A number of companies use ship-based measurement to profile icebergs. These methods comprise both above-water and below-water observations, yielding a full shape reconstruction of both the sail and keel of the iceberg. Figure \ref{fig:IcebergROV} shows an ROV operated by Fugro N.V. being deployed near an iceberg to obtain the underwater measurements for full iceberg profiling.

Above-water measurements include radar, lidar, and image-based photogrammetry. Lidar-based approaches can provide dense point cloud measurements of the iceberg's sail, yielding a very high resolution reconstruction. An added benefit of this sensing method is that the lidar point clouds have sufficient overlap in the ship's direction of travel to measure the iceberg's motion directly. This method is also used in ship-hull inspection, as in \cite{Papadopoulos2014} where the scan matching provided such accurate hull-relative odometry that the inertial measurements were essentially ignored, and the position estimate from the lidar was used to project sonar measurements to create the keel portion of the map.

To map the keel, multibeam or sidescan sonar is typically used. For smaller icebergs, the measurements can be made using sonars mounted to the ship. However, for very large icebergs, surface-based observation can be insufficient to measure deeper portions of the keel. Tethered ROVs or tow bodies outfitted with sonar can be used to access the lower parts, and even the bottom of the keel, but drive up the cost and complexity of the operation, and can put expensive assets at risk. 

The chief advantages of of ship-based survey are the ability to make simultaneous observations of the iceberg both above and below the surface, and the availability of GPS. The former allows the iceberg motion to be measured directly, leveraging the wider variety of sensing methods available above water, such as optical cameras and lidar, which simplifies the task of resolving the data into the moving reference frame. The availability of GPS signal above water also provides a reference for the observer's position in the world frame, which can aid in the estimate of the iceberg's trajectory over time. Disadvantages include risks involved in operating close to icebergs, and the inability to profile deep portions of the iceberg. 

\section{AUV-based iceberg profiling}

AUVs present several advantages over the above methods of profiling, but also face some unique challenges. AUVs are untethered, granting access to the full depth of the iceberg without risking the loss of an ROV should its tether get caught or severed by brash ice. Additionally, untethered AUVs can operate at higher speed than ROVs, yielding greater data coverage and shorter mission  time. 

An AUV can be programmed to fly around an iceberg at a desired standoff distance. Using a sideways-facing multibeam sonar, it sweeps out a wide swath of range data that can be used to recreate the shape of the iceberg, as illustrated in Figure \ref{fig:introSetup}. 

The challenges with AUV-based profiling are primarily navigational. The robot needs to be able to maneuver autonomously in a safe and robust manner around terrain that is moving, and resolve its sonar measurements into the moving frame of reference. Addressing this challenge is one of the primary motivations of this thesis.

\begin{figure}[!htb]
   \centering
   \includegraphics[width=\textwidth]{../graphics/sonarIllustrations/sweepMap_edited.png} % requires the graphicx package
   \caption{Illustration of an AUV profiling an iceberg keel. The vehicle circumnavigates the iceberg, sweeping out a point cloud of multibeam sonar measurements. }
   \label{fig:introSetup}
\end{figure}

\subsection{Nominal mission profile}
\label{sec.MissionProfileDef}
The work presented in this dissertation assumes a nominal mission profile where the vehicle follows a roughly helical path around an iceberg, slowly descending to observe deeper portions as it travels. This is illustrated in Figure \ref{fig.NominalProfile1}. This description is meant to provide context for the development to follow, but is not meant to indicate that this choice of vehicle and trajectory is \emph{unique}.  

The profiling process begins aboard a research vessel by locating a suitable iceberg and approaching to minimum safe distance. The AUV is then deployed in a direction to intercept the iceberg. Upon interception, the vehicle circumnavigates the iceberg in a counter-clockwise direction. It does this reactively, flying at a constant nominal speed, and using its obstacle avoidance sonar or DVL to maintain a constant standoff distance of 25 to 50 meters. In order to image the entire iceberg, the vehicle must fly at multiple depths, which is achieved with a small constant downward velocity, resulting in a helical trajectory.

\begin{figure}[htb]
   \centering
   \includegraphics[scale=.4]{../graphics/spiral.png} % requires the graphicx package
   \caption{Nominal spiral mapping trajectory }
   \label{fig:NominalProfile1}
\end{figure}

More details on the vehicle configuration, sensors, etc. can be found in Chapter \ref{ch.ProblemStatement}.

\subsection{Challenges with AUV profiling}

Profiling icebergs poses a set of unique challenges relative to other underwater mapping tasks. Specifically, icebergs can move a substantial amount during the time it takes to image their entire surface. Icebergs are free to translate in two directions, and rotate in heading. They are pushed at random by the wind  and currents, making it difficult to predict how they will move. (Additionally, they are prone to flip or ``turtle," pitching 180 degrees over very short time scales, though this dissertation does not address such dynamic corner cases of profiling.) 

If unaccounted for, this motion of the object being profiled will introduce warping and inconsistencies into the reconstruction. Figure \ref{fig:Challenge} shows the same data as Figure \ref{fig:idealMap}, but where the iceberg was in motion during the course of data collection. Since the iceberg was not stationary, simply reprojecting the range data based on the inertial (world-frame) trajectory (magenta) yields an inconsistent point cloud. However, if the vehicle's trajectory through the \emph{iceberg} frame of reference, rather than the world frame, can be determined, then the data can be reprojected into a self-consistent point cloud. This will be the guiding intuition behind the solution method presented in this thesis.

Unmodeled accelerations and rotations of the iceberg are indiscernible from navigation errors onboard the vehicle. The equivalence between errors due to motion and errors due to faulty inertial sensing is shown in detail in Chapter \ref{ch.IcebergGeometry}. Additionally, the non-inertial reference frame means that even infrastructure-based localization methods like GPS (if it were available underwater) and Ultra-short Baseline Interferometry (USBL) will appear to drift in the iceberg frame, since they report position in the inertial frame. In order for these localization methods to provide drift-free iceberg-relative positioning, infrastructure would need to be installed on the iceberg itself, which is prohibitive in most cases due to risk and reliability concerns.

\begin{figure}[!htb]
   \centering
   \includegraphics[width=.9\textwidth]{../graphics/ceresResults/SimData/RealIceberg1/rawNavNoLinks.png} % requires the graphicx package
   \caption{Simulated multibeam sonar point cloud with vehicle trajectory not corrected to account for iceberg motion.}
   \label{fig:Challenge}
\end{figure}

\subsection{Prior work in AUV-based profiling}

Previous efforts have investigated the use of AUVs for iceberg profiling. 
Zhou et. al. investigated using Slocum Glider vehicles to profile icebergs between Greenland and Newfoundland and Labrador in \cite{Zhou2014}. Icebergs are prevalent there, and pose risk to offshore oil installations and shipping. Gliders show promise as a low-cost data collection platform to aid in iceberg management, but their low power and control authority constraints limit their achievable resolution. However, cooperative efforts with autonomous surface vehicles show great potential for improving the profile resolution \cite{Smith2014}. The focus of that work has been developing a control strategy for the glider, given its motion constraints, that allows application to iceberg profiling. It assumes that the iceberg's motion can be adequately described purely with translation. This assumption is likely valid for small icebergs, which do not take long to profile, but breaks down for much larger icebergs, where drift has longer to accumulate. 

The problem of creating detailed profiles of free-drifting icebergs was addressed by Kimball in \cite{Kimball2011b}. There, he presented a method for creating high-resolution maps of free-drifting icebergs using AUVs based on SLAM principles. The method considered a single, constant-depth swath of sonar data collected by circumnavigating a free-drifting iceberg. It used measurements of the same area of the iceberg made at different times, often referred to as  ``loop closure" events, to estimate the iceberg's trajectory through inertial space during the course of the profiling mission and then subtract its motion from the reprojected data. This serves as a starting point for the work presented in this thesis with details on the similarities and differences between the two methods given in Chapters \ref{ch.LoopClosure}, \ref{ch.IcebergGeometry}, and \ref{ch.GraphSLAM}.

%%%%

%The method presented here extends the techniques of traditional AUV-based seafloor mapping to profiling icebergs. This approach requires that the vehicle's pose \emph{relative to the iceberg}, as opposed to the inertial pose, be determined in order to project the measurements into a self-consistent cloud. The iceberg's motion makes this challenging, as integrating inertial sensor measurements does not provide an accurate pose estimate in a non-inertial reference frame. Unmodeled accelerations and rotations of the iceberg are indiscernible from navigation errors onboard the vehicle: even with perfect inertial measurements, the vehicle would appear to drift in the iceberg's reference frame. The equivalence between errors due to motion and errors due to faulty inertial sensing is shown in detail in Chapter \ref{ch.IcebergGeometry}. Additionally, the non-inertial reference frame means that even infrastructure-based localization methods like GPS (if it were available underwater) and USBL will appear to drift in the iceberg frame, since they report position in the inertial frame. In order for these localization methods to provide drift-free iceberg-relative positioning, infrastructure would need to be installed on the iceberg itself, which is prohibitive in most cases due to risk and reliability concerns. 

%The apparent odometry error also complicates greatly the task of data correspondence between traversals of the same area of the iceberg, an important step for using environment-aided localization to generate a self-consistent reconstruction. This stems from the fact that most methods used to align point clouds of data like the ones generated by the multibeam assume that the clouds are rigid. The odometry uncertainty introduces flexibility into the point clouds, degrading the accuracy of the alignment algorithms. This will be discussed in detail in Chapter \ref{ch.LoopClosure}.

%As a final note, there do exist shape reconstruction techniques that do not rely on determining observer motion explicitly, which could circumvent the problem of determining iceberg-relative vehicle motion, but the sensing available to AUVs makes their application challenging. Methods like Structure from Motion (SFM) \cite{Sturm1996} or Scan Matching  \cite{Gutmann1996} use cameras or dense lidar to align measurements directly, taking advantage of sensing modalities that have large degrees of overlap between successive measurements, enabling comparison and alignment. The multibeam sonar has a very limited field of view in the direction of travel, so successive measurements do not typically overlap. This maximizes data coverage per distance traveled, but makes it impossible to perform scan matching for alignment of successive measurements. While the addition of other sensors might make these methods easier to apply, removing the need to estimate iceberg-relative pose directly, this would be an expensive solution. Chapter \ref{ch.IcebergGeometry} describes how an additional multibeam sonar could be used for this purpose, but for slow-moving bodies like icebergs, this addition is not essential. Chapter \ref{ch.Results} show that accurate maps can be generated using the same sensor suite as traditional bathymetry mapping.

\section{Contributions}

This dissertation presents a SLAM-based approach for generating accurate, self-consistent iceberg profiles using data collected from an AUV. It uses the method presented in \cite{Kimball2011b}, as a starting point, making several improvements and contributions: 

a) It provides a method for generating a self-consistent reconstruction of the \emph{entire} submerged portion of a free-drifting iceberg, including multiple swaths collected at different depths. 

b) It formulates the profiling problem entirely in iceberg-relative terms. The method described in \cite{Kimball2011b} worked in the inertial reference frame and solved for iceberg motion explicitly. Recasting the problem in the moving reference frame has operational benefits and also speeds up the reconstruction process by several orders of magnitude. 

c) It describes two tools for encoding loop closure information into the profiling process. The first is a graphical user interface with which a human user can detect these events. The second is a method for autonomous detection of loop closure events, drawing from well-developed work in the computer vision community.

d) It shows profiling results from simulation using shape data from a real iceberg, as well as results from AUV field data of an underwater canyon with simulated motion effects. In both cases, the algorithm is able to produce an accurate, self-consistent profile of the body despite large apparent odometry errors due to its motion.  

\section{Reader's guide}

\subsubsection*{Chapter \ref{ch.RelatedWork}: Related Work} A survey of methods currently used for iceberg profiling is presented, along with related work in Simultaneous Localization and Mapping, and methods for feature recognition and automatic loop closure detection.

\subsubsection*{Chapter \ref{ch.ProblemStatement}: Problem Definition} The vehicle and sensor characteristics used for this work are described, as well as assumptions about iceberg motion. Additionally, the SLAM framework as it pertains to iceberg profiling is described, along with motivation for using the GraphSLAM algorithm to scale the solution to multiple passes of data with many loop closure events.

\subsubsection*{Chapter \ref{ch.LoopClosure}: Methods for loop closure detection} Several methods to enable the detection of loop closure events are presented. Traditional approaches for doing this can be complicated by the presence of moving terrain and sensor limitations.

\subsubsection*{Chapter \ref{ch.IcebergGeometry}: Motion between loop closures}  The method developed in \cite{Kimball2011b} is recast into the moving reference frame of the iceberg. This reduces greatly the number of estimation parameters, and eliminates the need for the vehicle to have knowledge of its inertial position.

\subsubsection*{Chapter \ref{ch.GraphSLAM}: Solving for Iceberg Shape using GraphSLAM} The problem formulation developed in Chapter \ref{ch.IcebergGeometry} is expressed as a probabilistic graph and solved using the GraphSLAM method.

\subsubsection*{Chapter \ref{ch.Results}: Profiling results} Results are presented on simulated data of a drifting iceberg, as well as field data collected in Monterey Canyon. 

\subsubsection*{Chapter \ref{ch.FutureWork}: Summary and Future Work}
The work presented is summarized and possible avenues for future research are discussed.
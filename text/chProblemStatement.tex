% !TEX root = ../Thesis.tex

\chapter{Problem Statement}
\label{ch.ProblemStatement}

This chapter develops the equations governing the optimization problem used to estimate the unknown parameters in iceberg mapping. It uses the problem statement put forth by Kimball and Rock \cite{} as a starting point, but differs in two key regards. First, a number of mathematical substitutions are made to reduce the quantity of parameters to be estimated, and eliminate inertial terms from the optimization. Second, all derivatives are taken in the moving reference frame. These changes yield a smaller estimation problem with no sacrifice in accuracy, and remove the need for the vehicle to be supplied with inertial navigation fixes, which is an operational benefit.

Kimball's method had the structure shown in Figure \ref{fig:NestedLoop}. The optimization is split into linear and nonlinear parts. The inner loop estimation vector $\hat{\bar{x}}_{ls}$ comprises $O(10)$ iceberg motion spline control points $P_x$ and $O(1000)$ iceberg-fixed points, denoted $S_i$ in Figure \ref{fig:LoopEquation}.

\begin{align}
\hat{\bar{x}}_{ls} &= 
        \left[\begin{array}{c}
        \vec{P_x}\\
        \vec{S_1}\\
        \vdots\\
        \vec{S_n}
        \end{array}
        \right]
\end{align}

Although the inner loop is a linear operation, the number of parameters to be estimated grows linearly with the map size, and can result in slow performance for large icebergs. The method presented in this thesis eliminates the inner loop estimation entirely, resulting in an optimization were only the heading of the iceberg as a function of time need be estimated. The specifics of the manipulations used to eliminate this part of the optimization will be developed fully in the following sections. 

\begin{figure}[htbp]
   \centering
   \includegraphics[scale=.4]{../graphics/KimballIterativeOpt.png} % requires the graphicx package
   \caption{Iterative optimization structure employed by Kimball. The nonlinear heading parameters $\psi$ are chosen in the outer loop and fed to the inner loop as constants. $\hat{\bar{x}}_{ls}$ is a vector of O(10) motion parameters and O(1000) iceberg-fixed points to be estimated at each optimization step  \emph{Graphic courtesy of Peter Kimball}}
   \label{fig:NestedLoop}
\end{figure}

\begin{figure}[htbp]
   \centering
   \includegraphics[scale=.4]{../graphics/HammondIterativeOpt.png} % requires the graphicx package
   \caption{Iterative optimization structure employed by this thesis. The inner loop optimization has been replaced by a deterministic loop closure evaluation based on inertial navigation and DVL. The resulting optimization can be run several orders of magnitude faster than prior art, and does not suffer the same scalability issues.}
   \label{fig:NestedLoop2}
\end{figure}

\subsection{Vehicle and Sensor Characteristics}

The vehicle model used for this paper is similar to the Dorado-class AUV. This is a torpedo-shaped vessel that achieves control through vectored thrust. As such, the vehicle is turn-radius constrained. 

\begin{figure}[htbp]
   \centering
   \includegraphics[scale=.4]{../graphics/AllSonar.png} % requires the graphicx package
   \caption{Orientation of DVL and multibeam beams with respect to AUV.  \emph{Graphic courtesy of Peter Kimball}}
   \label{fig:DVL}
\end{figure}


This paper assumes that the AUV being used is equipped with a sideways-facing DVL such that all four beams are nominally incident on the side of the iceberg. With this configuration, the AUV maintains ``bottom lock" or ``iceberg lock" on the iceberg, giving it a measurement of its velocity in the iceberg's reference frame. The AUV is also assumed to have a multibeam sonar aligned with the scan plane perpendicular to the vehicle's nominal velocity, used for mapping. 

It is assumed that the AUV can measure depth through the use of absolute pressure sensors. This reduces the estimation of translation to a 2D-problem in $x$ and $y$.

For each sonar sounding from both the multibeam and DVL, a range is returned. Since the each beam's orientation in the vehicle frame is known, each measurement can be represented as a vector expressed in the vehicle's frame of reference.

The DVL returns velocity information as well. Specifically, it measures the frequency shift of the returned signal, which corresponds directly to the instantaneous rate of change of the length of the DVL beam $\left( \frac{d||\vec{r}^{\nicefrac{s_i}{V_{cm}}}||}{dt}\right)$. This is the component of the vehicle's velocity in the iceberg reference frame in the direction of the sonar beam. An important subtlety in this measurement is that the DVL only provides the velocity \emph{as expressed in the vehicle's frame of reference.} It provides no information about vehicle attitude or attitude rates. 

\section{Notation}

Before developing the equations, the notation used in this dissertaion is as follows:

\emph{Position vectors} are written as
\begin{equation*}
 \vec{r}^{~\nicefrac{A}{B}} 
\end{equation*}
where $A$ is the ``to" point and $B$ is the ``from" point.

\emph{Unit vectors} are denoted with a hat ($~\hat{}~$), and scalars are unadorned. Note: a hat can also indicate that the quantity is estimated. The meaning should be clear from context, especially when both hat and vector arrow are present.

% Velocity
\emph{Velocities} are written as 
\begin{equation*} 
~^{A}\vec{v}^{P} 
\end{equation*}
which should be read as ``the velocity of point $P$ in reference frame $A$."

% Angular Velocity
\emph{Angular velocities} are written as 
\begin{equation*}
 ~^{A}\vec{\omega}^{B}
\end{equation*}
which should be read as ``the angular velocity of reference frame $B$ in reference frame $A$."

Vector \emph{bases} will be denoted with right subscripts. For example $^I\vec{v}^P_{Q}$ is the velocity of point $P$ through reference frame $I$, \emph{expressed in} reference frame Q.   
% Derivatives
When taking \emph{derivatives}, left superscripts refer to the reference frame in which the derivative is being taken.
\begin{equation*}
 \frac{^{A}d(~)}{~dt}
\end{equation*}

Using this notation, \emph{vector differentiation} can be written as:

\begin{equation*}
 \frac{^{A}d(\vec{v})}{~dt} = \frac{^{B}d(\vec{v})}{~dt} + ~^{A}\vec{\omega}^{B}\times \vec{v}
\end{equation*}

Points are defined as:
\begin{align*}
O =&~ \text{Origin, fixed in inertial frame}\\
V_{cm} =&~ \text{Vehicle (Collocated with DVL)} \\
B_{cm} =&~ \text{Iceberg center of mass}\\
S_i =&~ \text{Sounding location (fixed in iceberg frame)}
\end{align*}
\\
\\
Reference frames are denoted using:
\begin{align*}
I =&~ \text{Inertial}\\
V =&~ \text{Vehicle} \\
B =&~ \text{Iceberg}
\end{align*}

%%%%%%%%%%%%%%%%%%
% end vector definitions
%%%%%%%%%%%%%%%%%%

\section{Geometric Problem Definition}

The starting point for the estimation problem is the so-called ``Loop Equation," as illustrated in Figure \ref{fig:LoopEquation}.

\begin{align}
\vec{r}^{\nicefrac{B_{cm}}{O}} + \vec{r}^{\nicefrac{S_i}{B_{cm}}} &= ~\vec{r}^{\nicefrac{V_{cm}}{O}} + \vec{r}^{\nicefrac{S_i}{V_{cm}}}
\end{align}

\begin{figure}[htbp]
   \centering
   \includegraphics[width=.8\textwidth]{../graphics/IcebergGeometry.png} % requires the graphicx package
   \caption{The geometry of iceberg mapping. }  
    \label{fig:LoopEquation}
\end{figure}

The loop equation gets its name from the fact that the two paths to get from $O$ to $S_i$ are equivalent, and define a ``loop."

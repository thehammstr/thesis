% !TEX root = ../Thesis.tex

\chapter{Mapping Results}
\label{ch.Results}

The following sections show GraphSLAM results from simulation and field data. 

\section{Data Collection}

This section details how data were collected for use in the GraphSLAM mapping algorithm.

\subsection{Iceberg Simulation}

A simulation environment was created in MATLAB to enable mapping results to be compared with known ``truth" for validation of the method. A screen capture from this envirionment is shown in Figure \ref{fig:IcebergSim}
 
 \begin{figure}[htbp]
   \centering
   \includegraphics[width=.7\textwidth]{../graphics/ceresResults/SimData/RealIceberg1/SimImage1.png} % requires the graphicx package
   \caption{Screen capture from the iceberg mapping simulation environment. The model was built from data collected from a real iceberg, part of the CNRC iceberg shapes and sightings database. The yellow cylinder represents the multibeam sonar-equipped AUV. The red contour is the current multibeam scan, and the cyan portion represents past scans. Returns are recorded up to the water line. The simulation allows the iceberg to be set adrift with arbitrary motion, and allows the reconstruction to be compared to truth data.}
   \label{fig:IcebergSim}
\end{figure}

A mesh was created based on 3D point cloud data of iceberg [XXX], obtained from the CNRC Iceberg Shapes and Sightings database. \cite{?} This particular iceberg was roughly 150 meters long at the waterline, with a keel depth of about 70 meters.

This mesh was set adrift with known translation and rotation. The simulated AUV then flew a helical path of increasing depth around the iceberg, in wall-following mode. Its feedback controller attempted to maintain a constant standoff distance as measured by the two forward-inclined DVL beams.

Multibeam sonar range returns were then generated via ray-tracing on the mesh. These measurements could be corrupted with random noise to simulate imperfect real-world sensing. 

When simulated at higher standoff distances, around 50 meters, the two forward DVL beams often lost contact with the iceberg surface. In these instances, the simulated DVL reported ``MAX RANGE" to the controller and was unable to estimate the full iceberg-relative velocity. To handle these incidents, the wall-following feedback controller would saturate its output to gracefully re-acquire the wall. In the mapping phase, the dropouts in measured velocity were interpolated between times when the vehicle had good velocity readings. The process noise during these outages was increased to reflect this. The AUV nominally flies at a constant velocity as expressed in the body frame, so these outages did not have a great effect on the end reconstruction, but a higher fidelity motion model could enable better odometry by using DVL velocity information when only a subset of the beams can see the iceberg surface. This concept will be discussed further in Chapter \ref{ch.FutureWork}.

\subsection{Soquel Canyon Wall Data}

[Talk to Rock to get full details on the mapping vehicle (and mention iceberg AUV?)]

\section{Loop Closure}

\subsection{Automatic Loop Closure}

The automatic loop closure detection scheme detailed in Chapter \ref{ch.LoopClosure} was applied to the Soquel Canyon wall dataset. In order to avoid introducing bad correspondences into the graph optimization, conservative thresholds were used on feature matching and outlier rejection. As a result, only two loop closure events were identified with high enough confidence to add them to the optimization. One of these matches is shown in Figure \ref{fig:autoLC}.

 \begin{figure}[htbp]
   \centering
   \includegraphics[width=.9\textwidth]{../graphics/siftMatches.png} % requires the graphicx package
   \caption{A loop closure event detected automatically by the method described in Chapter \ref{ch.LoopClosure}}
   \label{fig:autoLC}
\end{figure}

The resolution of the simulated iceberg model was insufficient for automatic loop closure detection. The simulated sonar returns were generated from a model that had already been smoothed from raw data, so there was not enough small-scale texture to detect pseudo-image features. Indeed, even with noise-free simulated measurements, the model's smoothness yielded worse point cloud alignment than the raw canyon data.

\subsection{Manually-assisted Loop Closure}

Loop closure events were encoded in semi-supervised fashion via the interactive graphical user interface, described in Section \ref{sec:GUI}. The simulation data optimization contained roughly 15 loop closure events, and the canyon data had around 25 events. In general, the more loop closure events the optimization contains, the better (more self-consistent) the results, but care must be taken not to supply the optimizer with bad loop closure information, as this will corrupt the results.

\section{Solution via Ceres}

The Ceres Library is a C++ open-source nonlinear least squares optimization library initially developed and used by Google. It has been used in a wide range of image processing and robotic mapping tasks by university and research groups around the world. Its flexibility and speed make it well-suited for these tasks, as, in general, no two experimental setups are exactly the same, and the problem size can be quite large.  

The library allows the user to define cost function objects that tie graph nodes to one another, and incrementally add them to the problem. 

The contributions discussed in Chapter \ref{ch.IcebergGeometry} made the application of this optimized solver a straightforward exercise. 

\subsection{Simulation Results}

\subsubsection{Thirty-meter standoff distance}

The 150 meter-long simulated iceberg's keel was reconstructed to an rms accuracy of 0.8 m. To compute this, the reconstructed shape was registered against the original model via ICP. For each point in the registered, reconstructed cloud, the distance to the model's nearest surface was recorded. These results are shown in Figure \ref{fig:RMS}

 \begin{figure}[htbp]
   \centering
   \includegraphics[width=.9\textwidth]{../graphics/ceresResults/SimData/RealIceberg1/RawNav.png} % requires the graphicx package
   \caption{The raw inertial navigation of the AUV during the simulated mapping run is shown by the pink contour. The iceberg's motion causes the large misalignments apparent in the projected multibeam measurements. The green line segments represent calculated loop closure events, provided in pre-processing using a semi-supervised scan matching method. In an accurate map, both ends of each segment should lie on the pink contour.}
   \label{fig:RawNav}
\end{figure}

 \begin{figure}[htbp]
   \centering
   \includegraphics[width=.9\textwidth]{../graphics/ceresResults/SimData/RealIceberg1/WithDVL1.png} % requires the graphicx package
   \caption{The incorporation of iceberg-facing DVL readings corrects for iceberg translation effects, but cannot resolve changes in the iceberg's heading. This results in a ``corkscrew effect" error pattern, where the reprojected multibeam data appear to twist about the iceberg center.}
   \label{fig:WithDVL1}
\end{figure}

 \begin{figure}[htbp]
   \centering
   \includegraphics[width=.9\textwidth]{../graphics/ceresResults/SimData/RealIceberg1/WithDVL2.png} % requires the graphicx package
   \caption{Top view of the map after incorporating DVL data to highlight the corkscrew error that remains due to iceberg heading change}
   \label{fig:WithDVL2}
\end{figure}

 \begin{figure}[htbp]
   \centering
   \includegraphics[width=.9\textwidth]{../graphics/ceresResults/SimData/RealIceberg1/FullSolution.png} % requires the graphicx package
   \caption{After processing the data with GraphSLAM, the remaining alignment error is eliminated. In the process of aligning the point clouds, dead-reckoned DVL errors are estimated along with the errors related to iceberg heading. The output is the the map and trajectory that minimizes the odometry and alignment errors, weighted by the relative uncertainty of each. }
   \label{fig:FullSimSol}
\end{figure}

 \begin{figure}[htbp]
   \centering
   \includegraphics[width=.9\textwidth]{../graphics/rmspoint8.png} % requires the graphicx package
   \caption{The reconstructed keel (blue points) laid over the model. The rms error was measured to be .8m for the roughly 150 meter-long iceberg. Red points mark the center of each face of the model, used to aid the calculation.}
   \label{fig:RMS}
\end{figure}

\subsubsection{Increased Standoff distance}

At an increased standoff distance of 50 meters, the DVL only remained locked in on the iceberg around 40\% of the time, yielding less precise odometry. Despite this, the mapping algorithm was able to correct for the iceberg's motion.

 \begin{figure}[htbp]
   \centering
   \includegraphics[width=.9\textwidth]{../graphics/ceresResults/SimData/RealIceberg1/50m_standoff/reconstruction_50m_standoff.png} % requires the graphicx package
   \caption{At 50m standoff distance, the vehicle was out of DVL velocity lock roughly 60\% of the time. In spite of this, the algorithm is able to recover and produce a self-consistent map.}
   \label{fig:FullSim50m}
\end{figure}

\subsection{Soquel Canyon Results}

\subsubsection{Automatic loop closure detection solution}

\subsubsection{Manual loop closure detection solution}

 \begin{figure}[htbp]
   \centering
   \includegraphics[width=.9\textwidth]{../graphics/ceresResults/CanyonData/BadDriftRealData.png} % requires the graphicx package
   \caption{Data collected by an AUV in a portion of Monterey Canyon with near-vertical walls. The navigation data was corrupted with a heading error to simulate the effects of iceberg motion or faulty heading sensor. This instance corresponds to a quite aggressive iceberg drift rate or a failed heading gyro. The resulting large dead reckoned drift complicates the task of detecting loop closure.}
   \label{fig:RealDataWithDrift}
\end{figure}

 \begin{figure}[htbp]
   \centering
   \includegraphics[width=.9\textwidth]{../graphics/ceresResults/CanyonData/ThreePassConvergence.png} % requires the graphicx package
   \caption{After processing with the Ceres solver, the large initial errors are eliminated. Three swaths of data collected at different depths are fused into one self-consistent map. }
   \label{fig:RealDataSolution1}
\end{figure}

 \begin{figure}[htbp]
   \centering
   \includegraphics[width=.9\textwidth]{../graphics/ceresResults/CanyonData/ThreePassConvergence1.png} % requires the graphicx package
   \caption{Detailed view of the SLAM reconstruction, highlighting the multiple swaths of data fused into a self-consistent map. As in the simulated data, the pink contour is the AUV trajectory estimate, and the green segments show pre-processed loop closure events. }
   \label{fig:RealDataSolution2}
\end{figure}

 \begin{figure}[htbp]
   \centering
   \includegraphics[width=.9\textwidth]{../graphics/ceresResults/CanyonData/ThreePassConvergence2.png} % requires the graphicx package
   \caption{Detailed view of the SLAM reconstruction, highlighting the multiple swaths of data fused into a self-consistent map. }
   \label{fig:RealDataSolution3}
\end{figure}
% !TEX root = ../Thesis.tex

\chapter{Mapping Results}
\label{ch.Results}

I has pretty result graphs

\section{Iceberg Simulation}

 \begin{figure}[htbp]
   \centering
   \includegraphics[width=.9\textwidth]{../graphics/ceresResults/SimData/RealIceberg1/SimImage1.png} % requires the graphicx package
   \caption{Screen capture from the iceberg mapping simulation environment. The model was built from data collected from a real iceberg, part of the CNRC iceberg shapes and sightings database. The yellow cylinder represents the multibeam sonar-equipped AUV. The red contour is the current multibeam scan, and the cyan portion represents past scans. Returns are recorded up to the water line. The simulation allows the iceberg to be set adrift with arbitrary motion, and allows the reconstruction to be compared to truth data.}
   \label{fig:IcebergSim}
\end{figure}

 \begin{figure}[htbp]
   \centering
   \includegraphics[width=.9\textwidth]{../graphics/ceresResults/SimData/RealIceberg1/RawNav.png} % requires the graphicx package
   \caption{The raw inertial navigation of the AUV during the simulated mapping run is shown by the pink contour. The iceberg's motion causes the large misalignments apparent in the projected multibeam measurements. The green line segments represent calculated loop closure events, provided in pre-processing using a semi-supervised scan matching method. In an accurate map, both ends of each segment should lie on the pink contour.}
   \label{fig:RawNav}
\end{figure}

 \begin{figure}[htbp]
   \centering
   \includegraphics[width=.9\textwidth]{../graphics/ceresResults/SimData/RealIceberg1/WithDVL1.png} % requires the graphicx package
   \caption{Incorporating iceberg-facing DVL readings corrects for iceberg translation effects, but cannot resolve changes in the iceberg's heading. This results in a ``corkscrew effect" error pattern, where the reprojected multibeam data appear to twist about the iceberg center.}
   \label{fig:WithDVL1}
\end{figure}

 \begin{figure}[htbp]
   \centering
   \includegraphics[width=.9\textwidth]{../graphics/ceresResults/SimData/RealIceberg1/WithDVL2.png} % requires the graphicx package
   \caption{Top view of the map after incorporating DVL data to highlight the corkscrew error that remains due to iceberg heading change}
   \label{fig:WithDVL2}
\end{figure}

 \begin{figure}[htbp]
   \centering
   \includegraphics[width=.9\textwidth]{../graphics/ceresResults/SimData/RealIceberg1/FullSolution.png} % requires the graphicx package
   \caption{After processing the data with GraphSLAM, the remaining alignment error is eliminated. In the process of aligning the point clouds, dead-reckoned DVL errors are estimated along with the errors related to iceberg heading. The output is the the map and trajectory that minimizes the odometry and alignment errors, weighted by the relative uncertainty of each. }
   \label{fig:FullSimSol}
\end{figure}

\section{Canyon Wall Data}

 \begin{figure}[htbp]
   \centering
   \includegraphics[width=.9\textwidth]{../graphics/ceresResults/CanyonData/BadDriftRealData.png} % requires the graphicx package
   \caption{Data collected by an AUV in a portion of Monterey Canyon with near-vertical walls. The navigation data was corrupted with a heading error to simulate the effects of iceberg motion or faulty heading sensor. This instance corresponds to a quite aggressive iceberg drift rate or a failed heading gyro. The resulting large dead reckoned drift complicates the task of detecting loop closure.}
   \label{fig:RealDataWithDrift}
\end{figure}

 \begin{figure}[htbp]
   \centering
   \includegraphics[width=.9\textwidth]{../graphics/ceresResults/CanyonData/ThreePassConvergence.png} % requires the graphicx package
   \caption{After processing with the Ceres solver, the large initial errors are eliminated. Three swaths of data collected at different depths are fused into one self-consistent map. }
   \label{fig:RealDataSolution1}
\end{figure}

 \begin{figure}[htbp]
   \centering
   \includegraphics[width=.9\textwidth]{../graphics/ceresResults/CanyonData/ThreePassConvergence1.png} % requires the graphicx package
   \caption{Detailed view of the SLAM reconstruction, highlighting the multiple swaths of data fused into a self-consistent map. As in the simulated data, the pink contour is the AUV trajectory estimate, and the green segments show pre-processed loop closure events. }
   \label{fig:RealDataSolution1}
\end{figure}

 \begin{figure}[htbp]
   \centering
   \includegraphics[width=.9\textwidth]{../graphics/ceresResults/CanyonData/ThreePassConvergence2.png} % requires the graphicx package
   \caption{Detailed view of the SLAM reconstruction, highlighting the multiple swaths of data fused into a self-consistent map. }
   \label{fig:RealDataSolution2}
\end{figure}
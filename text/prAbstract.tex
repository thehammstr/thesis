% !TEX root = ../thesis.tex

\prefacesection{Abstract}
\noindent An improved method for creating 3D reconstructions, or profiles, of the submerged portion of free-drifting icebergs using autonomous underwater vehicles (AUVs) and simultaneous localization and mapping (SLAM) is presented. These 3D reconstructions  provide context for scientific data collected around icebergs, and can aid in navigating back to points of interest on the iceberg for further study. Additionally, these reconstructions can be used to assess risk to infrastructure like oil rigs and pipelines. However, the motion of the iceberg during data collection complicates the mapping process by introducing warping and inconsistency into the reconstruction. This motion must be accounted for in order to obtain an accurate profile. 

The work presented here improves upon prior work in three areas. First, it enables profiling the full iceberg depth, including multiple swaths of data and many loop closure events. Second, it reformulates the motion estimation problem entirely in an iceberg-fixed frame of reference. This choice of reference frame yields computational speed increases of several orders of magnitude, and removes the need to provide an explicit estimate of the vehicle's world-relative position, which can be difficult to determine accurately. Third, it presents a method for automating the data correspondence problem, a crucial component of SLAM that is challenging in any underwater mapping task, and made even more so by the motion of the iceberg.

Both simulation and field results are presented.  For the simulation, a 150 meter long iceberg was modeled to an accuracy of 0.8m rms error. For the latter case, data recorded at an underwater sea cliff in Monterey Canyon was used as an analog for the vertical walls of a tabular Antarctic iceberg. The vehicle's navigation data was corrupted to simulate the effects of iceberg motion.
